\begin{table}[ht] \centering 
	\begin{threeparttable} \centering \caption{Displacement effects}\label{tab_soc_ext:reg_fe_assrate_displacement}
		{\def\sym#1{\ifmmode^{#1}\else\(^{#1}\)\fi} 
			\begin{tabular}{l*{6}{c}}
				\toprule 
				&\multicolumn{1}{c}{(1)}&\multicolumn{1}{c}{(2)}&\multicolumn{1}{c}{(3)}&\multicolumn{1}{c}{(4)}\\
				\midrule
				
% THIS TABLE WAS GENERATED BY ADDING ROWS OF 
%	1) assrate_neighbor_regions_fixed_effects
%	2) assrate_fixed_effects_leads_lags
\\
\textit{\textbf{Panel A: Spatial displacement}}\\
Game day            &       0.144         &       0.153         &       0.170         &       0.210         \\
                    &     (0.146)         &     (0.146)         &     (0.147)         &     (0.154)         \\
Effect size [\%]&        2.58         &        2.74         &        3.04         &        3.75         \\
Observations        &     960,848         &     960,848         &     960,848         &     960,848         \\

\\\\
\textit{\textbf{Panel B: Temporal displacement}}\\
Game day            &       2.777\sym{***}&       2.810\sym{***}&       2.857\sym{***}&       2.770\sym{***}\\
                    &     (0.346)         &     (0.338)         &     (0.336)         &     (0.300)         \\
Day after game      &      -0.105         &      -0.076         &      -0.065         &       0.279         \\
                    &     (0.189)         &     (0.190)         &     (0.192)         &     (0.182)         \\
Day before game     &       0.287         &       0.318         &       0.339         &       0.352\sym{*}  \\
                    &     (0.241)         &     (0.238)         &     (0.229)         &     (0.206)         \\
Effect size [\%]&       22.34         &       22.60         &       22.98         &       22.28         \\
Observations        &      87,438         &      87,438         &      87,438         &      87,438         \\
 \\\\
 Region FE           &         \checkmark         &         \checkmark         &         \checkmark         &         \checkmark         \\
 Date FE             &         \checkmark         &         \checkmark         &         \checkmark         &         \checkmark         \\
 Weather Controls    &           -         &         \checkmark         &         \checkmark         &         \checkmark         \\
 Holiday FE          &           -         &           -         &         \checkmark         &         \checkmark         \\
 Interact FE         &           -         &           -         &           -         &         \checkmark         \\
 
				\bottomrule 
		\end{tabular}}
		\begin{tablenotes} 
			\item \scriptsize \emph{Notes:} Estimates are based on the model shown in equation \ref{eq_soc_ext:model}. Panel A contains specifications that use daily data (excluding June) spanning the time window 2011-2015 for regions that share a border with a distract in which a stadium is located. Panel B shows specifications that use daily data (excluding June) spanning the time window 2011-2015 for regions that host games of a football team from the top three leagues. The outcome variable is defined as the number of assaults per million population. Population-weighted coefficients show the change in the outcome variable due to a home game. Days are defined to run from 6:00\textsc{am} until 5:59\textsc{am} the following day to accommodate the fact that offenses committed in the early morning hours have their origin in the preceding day. The effect size corresponds to the percent change of the assault rate due to a football game in relation to the mean when no game takes place. The estimates are based on the model shown in equation \ref{eq_soc_ext:model} and use the same set of controls as column 4 of Table \ref{tab_soc_ext:reg_fe_assrate} (including region and date fixed effects, their interactions as well as holiday and weather controls). See Table \ref{tab_soc_ext:reg_fe_assrate} for additional details. Two-way clustered standard errors at region-year and year-month level are reported in parentheses. \newline Significance levels: * p < 0.10, ** p < 0.05, *** p < 0.01.
		\end{tablenotes} 
	\end{threeparttable} 
\end{table}

% original notes
% Estimates are based on the model shown in equation \ref{eq_soc_ext:model}. Panel A contains specifications that use daily data (excluding June) spanning the time window 2011-2015 for regions that share a common boundary with a distract that contains a stadium of a football team from the first three leagues of the German football league system. Panel B shows specifications that use daily data (excluding June) spanning the time window 2011-2015 for regions that host games of a football team from the first three leagues of the German football league system.	The outcome variable is defined as the number of assaults per million population. Population-weighted coefficients show the change in the outcome variable due to a home game. Days are defined to run from 6:00 AM until 5:59 AM the following day to accommodate the fact that offenses committed in the early hours have their origin in the preceding day. The effect size corresponds to the percent change of the assault rate due to a football game in relation to the mean when no game takes place. Control variables shown as \textit{Date FE} include dummies for day-of-week, month, and year. \textit{Weather controls} include air temperature (average, maximum, and minimum), minimum ground temperature, vapor pressure, air pressure, cloud cover, air humidity, precipitation, sun shine duration, snow depth and wind velocity. \textit{Holiday FE} is a dummy variable for public and school holidays, as well as for other peculiar days. Control variables shown as \textit{Interact FE} consist of interactions of region dummies with all elements of the date fixed effects. Two-way clustered standard errors at region-year and year-month level are reported in parentheses. \newline Significance levels: * p < 0.10, ** p < 0.05, *** p < 0.01.