\begin{table}[ht] \centering 
	\begin{threeparttable} \centering \caption{Effects on assault rate}\label{tab_soc_ext:reg_fe_assrate}
		{\def\sym#1{\ifmmode^{#1}\else\(^{#1}\)\fi} 
			\begin{tabular}{l*{6}{c}}
				\toprule 
				&\multicolumn{1}{c}{(1)}&\multicolumn{1}{c}{(2)}&\multicolumn{1}{c}{(3)}&\multicolumn{1}{c}{(4)}\\
				\midrule
				Game day            &       2.740\sym{***}&       2.766\sym{***}&       2.813\sym{***}&       2.677\sym{***}\\
                    &     (0.319)         &     (0.312)         &     (0.313)         &     (0.284)         \\
\midrule Effect size [\%]&       22.00         &       22.21         &       22.59         &       21.50         \\
Observations        &      88,028         &      88,028         &      88,028         &      88,028         \\
Region FE           &         yes         &         yes         &         yes         &         yes         \\
Time Fe             &         yes         &         yes         &         yes         &         yes         \\
Weather Controls    &           -         &         yes         &         yes         &         yes         \\
Holiday FE          &           -         &           -         &         yes         &         yes         \\
Interact FE         &           -         &           -         &           -         &         yes         \\
 
				\bottomrule 
		\end{tabular}}
		\begin{tablenotes} 
			\item \scriptsize \emph{Notes:} Estimates are based on the model shown in equation \ref{eq_soc_ext:model}. The specifications use daily data (excluding June) spanning the time window 2011-2015 for regions that host games of a football team from the first three leagues of the German football league system. The outcome variable is defined as the number of assaults per million population. Population-weighted coefficients show the change in the outcome variable due to a home game. Days are defined to run from 6:00 AM until 5:59 AM the following day to accommodate the fact that offenses committed in the early hours have their origin in the preceding day. The effect size corresponds to the percent change of the assault rate due to a football game in relation to the mean when no game takes place. Control variables shown as \textit{Date FE} include dummies for day-of-week, month, and year. \textit{Weather controls} include air temperature (average, maximum, and minimum), minimum ground temperature, vapor pressure, air pressure, cloud cover, air humidity, precipitation, sun shine duration, snow depth and wind velocity. \textit{Holiday FE} is a dummy variable for public and school holidays, as well as for other peculiar days. Control variables shown as \textit{Interact FE} consist of interactions of region dummies with all elements of the date fixed effects. Two-way clustered standard errors at region-year and year-month level are reported in parentheses. \newline Significance levels: * p < 0.10, ** p < 0.05, *** p < 0.01.
		\end{tablenotes} 
	\end{threeparttable} 
\end{table}