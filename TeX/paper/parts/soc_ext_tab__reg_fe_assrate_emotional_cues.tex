\begin{table}[H] \centering 
	\begin{threeparttable} \centering \caption{Effect of emotional cues}\label{tab_soc_ext:reg_fe_assrate_emotional_cues}
		{\def\sym#1{\ifmmode^{#1}\else\(^{#1}\)\fi} 
			\begin{tabular}{l*{3}{c}}
				\toprule 
				&\multicolumn{1}{c}{(1)}&\multicolumn{1}{c}{(2)}&\multicolumn{1}{c}{(3)}\\
				&&\multicolumn{2}{c}{Card \& Dahl (2011) specification}\\
				\cmidrule(lr){3-4}
				&\clb{c}{Upset\\event index} & \clb{c}{predicted\\outcomes}& \clb{c}{predicted\\and actual\\outcomes}\\
				\midrule
				Upset event (Index) &       2.717\sym{***}&                     &                     \\
                    &     (0.406)         &                     &                     \\
No upset event (Index)&       2.655\sym{***}&                     &                     \\
                    &     (0.289)         &                     &                     \\
Expected to lose    &                     &       3.495\sym{***}&       3.376\sym{***}\\
                    &                     &     (0.722)         &     (0.836)         \\
Expected to win     &                     &       2.752\sym{***}&       2.907\sym{***}\\
                    &                     &     (0.366)         &     (0.356)         \\
Expected to be close&                     &       2.437\sym{***}&       2.369\sym{***}\\
                    &                     &     (0.321)         &     (0.383)         \\
Expected to lose and won (upset win)&                     &                     &       0.599         \\
                    &                     &                     &     (1.554)         \\
Expected to be close and lost (upset loss)&                     &                     &       0.194         \\
                    &                     &                     &     (0.610)         \\
Expected to win and lost (upset loss)&                     &                     &      -0.792         \\
                    &                     &                     &     (0.667)         \\
\midrule Observations&      88,028         &      88,028         &      88,028         \\
Region FE           &         yes         &         yes         &         yes         \\
Time Fe             &         yes         &         yes         &         yes         \\
Weather Controls    &         yes         &         yes         &         yes         \\
Holiday FE          &         yes         &         yes         &         yes         \\
Interact FE         &         yes         &         yes         &         yes         \\
 
				\bottomrule 
		\end{tabular}}
		\begin{tablenotes} 
			\item \scriptsize \emph{Notes:} Estimates are based on the model shown in equation \ref{eq_soc_ext:model}. The gameday indicator is replaced by an index that captures unsettling events. The specifications use daily data (excluding June) spanning the time window 2011-2015 for regions that host games of a soccer team from the first three leagues of the German football league system. The outcome variable is defined as the number of assaults per million population. Population-weighted coefficients show the change in the outcome variable due to a home game. Days are defined to run from 6:00 AM until 5:59 AM the following day to accommodate the fact that offenses committed in the early hours have their origin in the preceding day. Control variables shown as \textit{Date FE} include dummies for day-of-week, month, and year. \textit{Weather controls} include air temperature (average, maximum, and minimum), minimum ground temperature, vapor pressure, air pressure, cloud cover, air humidity, precipitation, sun shine duration, snow depth and wind velocity. \textit{Holiday FE} is a dummy variable for public and school holidays, as well as for other peculiar days. Control variables shown as \textit{Interact FE} consist of interactions of region dummies with all elements of the date fixed effects. Two-way clustered standard errors at region-year and year-month level are reported in parentheses. \newline Significance levels: * p < 0.10, ** p < 0.05, *** p < 0.01. \newline 	The upset event index in column 1 is defined as a dummy variable equal to one if one of the following events take place: a penalty is awarded (20\% of all games), a red card is being issued (10\% of all games), or the referee receives a non-sufficient grade (15\% of all games). In columns 2 and 3 I use data from oddsportal.com to classify games as expected to win/lose/be close.
		\end{tablenotes} 
	\end{threeparttable} 
\end{table}