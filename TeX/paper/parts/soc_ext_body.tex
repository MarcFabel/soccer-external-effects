
% introductory paragraph
\lettrine[lines=2,nindent=0pt]{\color{darkgray}\textbf{F}}{ootball} has grown from being a favorite sport in many parts of the world into a respected industry, which is reflected by high attendance rates and considerable revenues.\footnote{Except where otherwise noted, football refers to the sport also known as soccer.} For instance, the \textit{Bundesliga}, the German top league, is a major reason for public mass gatherings and had with 42,700 spectators the highest average attendance per match of all European leagues in the season 2017/18. Also in economic terms, football is synonymous with big business as the \textit{Bundesliga} ranks second in Europe with generated revenues of 3.17 billion euros \citep{deloitte2019report}. Due to their great popularity, professional football matches generate external effects in many regards. On the positive side, there is higher consumer spending (e.g. merchandising, catering, and accommodation) and ultimately increased local tax revenues. On the negative side, the games induce air and noise pollution, and a higher frequency of accidents, among other things. Violent fan behavior is a particularly pervasive aspect in this context and affects individuals' health, the penal system, and the police. In the 2017/18 season, the State and Federal Police force invested over 2.11 million working hours to ensure the spectators' safety at professional football matches in Germany \citep{zis17_18}. There has been a widely noticed legal dispute between the Federal State of Bremen and the German Football League (DFL) regarding the reimbursement of costs for high-risk games, in which the \cite{fac_2019} declared the charges imposed by Bremen as legitimate. This study does not only focus on football matches where social interactions lead to violence, but also deals with the question of whether the negative externalities of football games that bring so much prosperity should be absorbed. 




% STRATEGY & DATA
The aim of this paper is to examine in detail the effects of football games on violent behavior in the German professional football league. To assess the causal impact of football matches on physical assaults, I employ a generalized difference-in-differences approach that exploits the variation in the timing of the football games. I compare the level of assaults on days with and without home games while controlling for any potential source of heterogeneity across days of the week, month, and year, and taking into account other possible confounding variation from weather or holidays. In order to use the time-series and cross-sectional variation in the data, I construct a panel at the municipality-day level. I collect and match data from various sources to analyze the effect of 4,461 football games in the period 2011-2015. First, detailed information on all matches in the first three leagues is obtained via web scraping.\footnote{This includes, but is by no means limited to the time and location of the games. Furthermore, I exploit betting odds that reflect pregame expectations.} Second, the primary outcome variable, the rate of physical assaults, is derived from comprehensive registry data provided by the Federal Criminal Police Office. It covers all individuals that become victims of a crime against their legally protected rights in the sample regions. Third, data from weather monitors and a time-series of holidays at the state level are matched to the data to account for possible confounders. Last, population figures from the Federal Statistical Office are merged to construct regression weights and create the outcome variable. 



% findings
I find significant and robust evidence that football matches lead to an increase in the prevalence of physical assaults. A home game increases the assault rate by on average 21.5 percent. The effects are almost entirely driven by male victims and are most pronounced in the 18-29 age group. Moreover, a heterogeneity analysis reveals that the effects are caused by attacks on victims with no prior relationship to the suspect and completed offenses (i.e. no attempts). Assaults on police officers on duty account for 16 percent of the baseline effect of the entire sample. I find no evidence that the effects of football games on the prevalence of assaults are offset by reductions in violent acts on days adjacent to the game day or in nearby areas. The focus on home games in the empirical framework does not pose a threat to the identification strategy, as away games do not change the level of physical assaults in the home district.



% channels 
There are a number of theories rationalizing spectator violence. For instance, the frustration-aggression hypothesis, proposed by \cite{dollard1939frustration}, interprets violent behavior as an act to rehabilitate self-esteem triggered by a frustrating event, such as the defeat of one's favorite team. Another prominent theory is the idea of social learning, according to which spectators mimic the behavior displayed on the field \citep{bandura1973aggression, bandura2007social}. In my analysis, I test whether the frustration-aggression hypothesis can serve as an explanation for the increase in physical assaults. To do so, I use emotional cues from either emotionally unsettling events during the game or when game outcomes do not align with pregame expectations. In both cases, there is no evidence to support the frustration-aggression hypothesis. In the second step, I consider the prominence of games and teams as potential channels. In games with high rivalry, the impact on the rate of physical assaults is three times higher. This larger effect may be related to the mutual dislike of opposing fan groups or to the social learning theory, as high-rivalry matches are characterized by more intensity on the field. Furthermore, I find larger effects for the highest division of the league system, which attracts the most viewers. Ultimately, I cannot pinpoint the one channel at play, but there is suggestive evidence that spectators mimic the behavior of the players, that violent viewers select into specific matches, and eventually a mechanical agglomeration effect. 



% back of envelope calculations
Back-of-the-envelope calculations based on my estimates indicate that the crime externalities of football games are substantial. In the season of 2014/15, for example, football games in the top three leagues of the German football league system explain 17.7 percent of assault reports and lead to 18,770 additional assaults in the sample regions. \cite{glaubitz2016kostet} estimate the social cost of an assault at 5,067 euros (in 2020 prices), which implies an annual social cost of 95 million euros.\footnote{The estimated social costs of \cite{glaubitz2016kostet} are conservative compared to estimates from other countries. In the United States, the social costs of an assault range from 17,300 to 68,000 euros \citep{miller1996victim,cohen2004willingness}. In New Zealand the social costs are estimated at 6,400 euros \citep{roper2006estimating}, and in Great Britain the estimated costs are 2,300 euros \citep{dubourg2005economic}. All costs are in 2014 prices.}




% literature
This paper complements the literature on sporting events and their effects on different dimensions of crime. Studies in the US American context demonstrate that American college football games lead to a higher incidence of rape \citep{lindo2018college}, physical assaults, vandalism, disorderly conducts, and alcohol-related offenses \citep{rees2009college}. Furthermore, it has been documented that intimate partner violence is linked to unsettling defeats in professional American football games \citep{card2011family}. In the European setting, studies have shown that professional football games increase property and violent crimes \citep{marie2016police}, and shift offenses both temporally \citep{montolio2016time} as well as spatially \citep{montolio2019measuring}.




% extension literature
The empirical analysis in this paper contributes to previous literature in many regards. I compile a unique data set from various sources to comprehensively characterize the impact of football matches on criminal behavior in the top three German professional football divisions. Using the universe of criminal offenses, I show that almost one in five physical assaults can be attributed to professional football games. In addition to the overall effect, I show effect heterogeneity by victim and crime characteristics. When investigating channels, I find no support for the frustration-aggression hypothesis, but evidence for the social learning theory and a selection story in which violent fans self-select into specific matches.
% emphasize more the importance of football in Germany





% roadmap
The remainder of the paper is structured as follows. The next section provides information about football games, their relationship with violent spectator behavior, and previous literature. Section \ref{sec_soc_ext:data} explains the data and the variables. Section \ref{sec_soc_ext:empirical_strategy} contains a description of the empirical framework. Section \ref{sec_soc_ext:results} reports results, validity checks, a discussion on potential channels, and robustness tests. Section \ref{sec_soc_ext:conclusion} concludes.






%--------------------------------------------------------------------
% BACKGROUND
%--------------------------------------------------------------------
\bigskip
\section{Background}\label{sec_soc_ext:background}



% German Football League System
\subsection{The German Football League System}
The three fully professional divisions in the German football league system are managed under the jurisdiction of the German Football Association (DFB) and the German Football League (DFL). While the top two leagues, \textit{Bundesliga} and \textit{2. Bundesliga}, are organized by the DFL, the third division, \textit{3. Liga}, is run by the DFB itself. Teams can be promoted or relegated from one league to another. The top two divisions consist of 18 teams playing 17 home and away games in one season. The third league contains 20 teams playing 19 home and away games.

The empirical approach of this paper exploits the variation in the scheduling of matches. Since 2006, the match schedules for the \textit{Bundesliga} and the \textit{2. Bundesliga} are created with a software that uses integer linear programming.\footnote{For details on how the match schedules are created, please refer to \href{https://www.bundesliga.com/de/bundesliga/news/so-entsteht-der-spielplan-bundesliga-2-bundesliga-dfl-computer-software.jsp}{\texttt{https://www.bundesliga.com}}.} The software outlines the rough details such as the matches per gameday. The exact date and time, however, are determined in the course of the season. The later exact scheduling makes it possible to take into account guidelines from local authorities, security bodies, the Central Sports Intelligence Unit (ZIS), international football associations (FIFA/UEFA), fans, clubs, and stadium operators. In addition to obvious restrictions such as the fact that home games of neighboring clubs should be scheduled at different times, the DFL has to consider public holidays, other major events, or match dates of international competitions.



% Soccer and Crime
\subsection{Football and Violent Crime}

% ULTRAS & HOOLIGANS
Spectator violence has a long tradition in the context of professional football in Germany. The change in names for football fans illustrates that spectator behavior has changed considerably \citep{pilz2005kuttenfan}. In the 1960s and 1970s, the peaceful fan base was refereed to as \textit{camp-followers}, while one decade later the first problems of spectator violence emerged with the so-called \textit{football rowdies}. In the 1980s, spectator violence was omnipresent, mainly due to the hooligan movement. Since the late 1990s, a new group has appeared in the stadiums, the \textit{ultras}. Originally from Italy, the \textit{ultras}are dedicated to fighting the commercialization of football and to revitalizing traditional football culture. Over the last years, the number of violent fans has been increasing. The police distinguish between three types of football fans. Category A includes peaceful fans, category B consists of fans inclined to violence, and category C contains fans who actively seek violence (violent criminals). Originally, the \textit{ultras} were predominantly assigned to category A and occasionally to category B. Recently, however, a substantial share of the \textit{ultras} has been classified as members of categories B and C. In the season 2017/18, for instance, the \cite{zis17_18} identified 13.633 individuals in the top three leagues who were either prone to violence or seeking violence. With more potential agitators, police officers report more aggression directed at them \citep{feltes2010fussballgewalt}. However, it is not only the growing number of violent fans that is a problem, but also the fact that the event character of violence during football games is increasing \citep{pilz2005kuttenfan}. Violent actions are less and less connected with the events on the playing field. In fact, violence serves as an opportunity for identification in the search for one's identity. Some violent fans find self-assertion by joining like-minded individuals to discover their strength. The question is how to address the challenges posed by \textit{hooligans} and \textit{ultras}. \cite{poutvaara2009police} investigate the impact of anti-hooliganism policy in the Swedish context. They find that the assignment of other duties to the Sport Intelligence and Tactical Unit was accompanied by an increase in violent crime by hooligans. \cite{feltes2010fussballgewalt} emphasizes the concept of `balanced policing', which acknowledges that risks from spectators are dynamic and require quick responses to either escalate or deescalate.\footnote{Depending on the types of violent fans the police are confronted with, the officers have to adjust their behavior accordingly \citep{feltes2010fussballgewalt}. On the one hand, a zero-tolerance policy must be applied to \textit{hooligans} and any failure to take drastic measures is seen as an invitation to act out violent desires. On the other hand, \textit{ultras} should be given some leeway to allow self-regulation.} 



% THEORIES OF SPECTATOR VIOLENCE
There are a number of theories rationalizing spectator violence.\footnote{Theoretically, football games can also lead to a decrease in the assault rate. This may happen if individuals with a higher propensity for violent behavior attend the matches. \cite{dahl2009movie} find a similar effect for violent movies and refer to this as the `self-incapacitation effect'. However, this effect can only be demonstrated with hourly data. As my analysis draws on daily data, I focus on theories as to why the rate of violence increases due to football games.} First, the higher number of assaults could be a merely mechanical effect resulting from increased social contacts and a change in the way interactions take place \citep{montolio2019measuring}. For instance, the \textit{Bundesliga} has the highest average number of spectators per match of all European major football leagues \citep{wicker2017effect}. In the 2014/15 season, an average of more than 42,000 spectators attended a game of the highest division of the German league system. This aspect is particularly relevant for smaller municipalities, which attract thousands of spectators per game \citep{lindo2018college}. Second, spectator violence may be the result of physiological arousal \citep{branscombe1992role}. Potential determinants of physiological arousal include increased heart-rate, blood pressure, and respiration, which may influence affect, cognition, and (anti-)social behavior. Third, \cite{dollard1939frustration} propose the idea of the frustration-aggression hypothesis.\footnote{\cite{berkowitz1989frustration} provides a helpful overview of the frustration-aggression hypothesis.} If the favorite team suffers a defeat, this may precipitate aggressive fan behavior (see, for instance, the results of \cite{card2011family}). Fourth, \cite{bandura1973aggression, bandura2007social} postulates the notion of `social learning' or mimetic behavior. Regardless of the outcome of the match, the mere observation of the game suffices to trigger violent actions.
%Second, instinctual theories of aggression find human aggression genetically based \citep{baron1977human}. In other words, aggressive behavior is an inevitable predisposition. 



% THE ROLE OF ALCOHOL
Alcohol plays a crucial role in the context of spectator violence in German football stadiums. \cite{cook2013virtuous} describes the pharmacological effects of alcohol consumption on aggression and cognitive functions. Alcohol consumption is associated with a loss of inhibition and impaired judgment. Furthermore, experiments have shown that participants exhibit more aggressive behavior after drinking. The 2018 edition of the \citeauthor{PCS2018} specifies that more than one in four assaults (26.2 percent) were committed under the influence of alcohol. While in some countries alcoholic beverages are prohibited on the premises (e.g. in Brazil since 2003), the rules in German stadiums are somewhat ambiguous. The DFB's security guidelines stipulate that the sale of alcoholic beverages is forbidden before and during games in the stadium. Nevertheless, with the approval of the responsible local security bodies, the hosting clubs can deviate from the regulations, on their own responsibility. Only in the case of high-risk games, the clubs are urged to comply with the ban.\footnote{For details, please refer to: \href{https://www.sueddeutsche.de/sport/fussball-alkohol-in-stadien-nur-mit-ausdruecklicher-einwilligung-dpa.urn-newsml-dpa-com-20090101-150625-99-01743}{\texttt{https://www.sueddeutsche.de}}.} The clubs, however, have a strong incentive to deviate from the ban as more than one-sixth (538 million Euros in the 2017/18 season) of the \textit{Bundesliga} clubs' earnings are generated by matchday revenues (e.g. tickets and catering) and the sale of alcoholic beverages is a substantial part of this \citep{deloitte2019report}. For this reason, alcohol and its potential side effects are very present in German football arenas.
% any data on how much is consumed? https://www.news.de/sport/855220143/schalke-hat-die-durstigsten-fans/1/
% Statista survey: 56 percent of respondents of a panel survey were in favor of banning alcohol from football stadiums, 35 percent defended the right to drink alcohol in football stadiums: https://de.statista.com/statistik/daten/studie/548802/umfrage/verbot-von-alkohol-in-fussballstadien/


\subsection{Previous Literature}
% Literature
The paper relates to previous literature that investigates the impact of large scale sporting events on various types of criminal behavior. Studies in the US American context often use offense reports from the National Incident Based Reporting System to investigate the impact of American (college) football games on crime. \cite{rees2009college} exploit within agency variation to study the effects of Division I-A college American football games on various offense categories for the years 2000-2005.\footnote{Almost all of the following studies exploit within law enforcement agency variation over time while controlling for weather, holidays and other sources of heterogeneity over time.} They find an increase in assaults, vandalism, arrests for disorderly conduct, and alcohol-related offenses in hosting municipalities on game days. Larger effects are associated with unexpected game outcomes, defined as when lower ranked teams win against higher ranked teams. The effects of away games are not significantly different from zero. \cite{lindo2018college} examine the effect of college party culture in the context of Division 1 American football games on sexual assaults. They show that the daily reports of rape victimization among 17-24-year-old women increase by 28 percent on games days. The effects are larger for home games, prominent matches, and games involving teams playing in the better ranked subdivision of Division 1. Furthermore, they show that game outcomes matter: unexpected wins lead to a strong increase in the number of rapes. \cite{card2011family} analyze the impact of emotionally unsettling events associated with wins and losses of professional American football teams on family violence for the years 1995-2006. They find a ten percent increase in intimate partner violence in the event of unexpected losses (when the home team was expected to win). There are no effects for unexpected wins or when the game expectations predict a close match. The effects of intimate partner violence of men against women are pronounced around the end of the game and are larger for higher-profile games. 

In the European context, researchers focus on football matches to estimate the effect of large scale sporting events on crime. \cite{marie2016police} investigates the effect of football matches on crime in London using hourly offense data from the Metropolitan Crime Statistics System. His results show that property crimes increase (decrease) by 4 percent (3 percent) for every additional 10,000 spectators attending a home (away) game. Violent crimes are only affected by derby matches. \cite{montolio2016time} study the temporal impact of football matches on criminal behavior in Barcelona (2007-2011). They match reports of registered crime with football matches played by the Football Club Barcelona (FCB) to see whether the games lead to temporal shifts in criminal activity. They employ a panel approach comparing crime rates during the same time window on the same day of the week with and without FCB games. Their results indicate temporal shifts for criminal activities of thefts, criminal damage, robberies, and gender violence. Moreover, instances of gender violence increase after home defeats. In a follow-up study, \cite{montolio2019measuring} investigate the spatial dimensions of crime externalities associated with football games. Their findings show that, in the event of a home game, theft rates (mainly pickpocketing) increase in the entire city. The impact is larger for regions in close proximity to the stadium. During away games, by contrast, the number of thefts occurring near the stadium is relatively smaller. The effects of football matches on assaults are analogous to thefts. In fact, on days with home matches, there are relatively more physical assaults in areas close to the stadium.
%\footnote{In addition to the papers presented in this section, other noteworthy studies in that strand of research are: \cite{kalist2016national}.}





%--------------------------------------------------------------------
% DATA & VARIABLES
%--------------------------------------------------------------------
\bigskip
\section{Data}\label{sec_soc_ext:data} 
% Description data
The data set used for the analysis covers the time window from 2011 to 2015 and contains regions in which professional football games take place. The analysis is conducted at the municipality level, the smallest territorial division in Germany. I combine various data sources to examine the impact of professional football games on violent behavior.

% crime - assault distribution over time
% Figiure from Stata
\begin{figure}[ht]\centering
	\caption{Distribution of assaults across time}\label{fig_soc_ext:assault_time_distribution_2014}

	\includegraphics[width=\linewidth]{descriptive/soc_ect_desc_crime_ass_time.pdf}

	\begin{minipage}{\linewidth}
		\scriptsize{\emph{Notes:} The figure shows the distribution of assaults across the course of a day, days of the week, across months (adjusted for the number of days per month), and across the days of the year in the Federal Republic of Germany.\newline \emph{Source:} Own representation with data from the Federal Criminal Police Office.\newline \hspace{1 em} $^1$ The figures in panel C and D are solely based on the year 2014. Please consult the appendix for figures from the other years.}
	\end{minipage}
\end{figure}


% FIGURES FROM PYTHON
%\begin{figure}[ht]\centering
%	\caption{Distribution of assaults across time}\label{fig_soc_ext:assault_time_distribution_2014}
%	\begin{subfigure}[h]{0.48\linewidth}\centering
%		\includegraphics[width=\linewidth]{descriptive/soc_ext_assaults_per_hour.pdf}
%	\end{subfigure}
%	\begin{subfigure}[h]{0.48\linewidth}\centering
%		\includegraphics[width=\linewidth]{descriptive/soc_ext_assaults_per_dow.pdf}
%	\end{subfigure}
%
%	\begin{subfigure}[h]{0.48\linewidth}\centering
%		\includegraphics[width=\linewidth]{descriptive/soc_ext_assaults_per_month_2014.pdf}
%	\end{subfigure}
%	\begin{subfigure}[h]{0.48\linewidth}\centering
%		\includegraphics[width=\linewidth]{descriptive/soc_ext_assaults_per_day_2014_single.pdf}
%	\end{subfigure}
%	\begin{minipage}{\linewidth}
%		\scriptsize{\emph{Notes:} The figure shows the distribution of assaults across the course of a day, days of the week, across months (adjusted for the number of days per month), and across the days of the year in the Federal Republic of Germany.\newline \emph{Source:} Own representation with data from the Federal Criminal Police Office.\newline \hspace{1 em} $^1$ The figures in panel C and D are solely based on the year 2014. Please consult the appendix for figures from the other years.}
%	\end{minipage}
%\end{figure}


\subsection{Crime Data}
The crime data is derived from the German Police Crime Statistics, which is provided by the Federal Criminal Police Office.\footnote{Many aspects of the data preparation are inspired by \cite{hener2019noise} who uses the same data to examine the causal effect of noise pollution on criminal activities.} It includes the universe of individuals who were victim to a crime against their legally protected personal rights between 2011 and 2015. However, as the data is not reported until after police procedures are completed, only data from January 2011 to May 2015 is used to avoid problems with lags between the occurrence of the crime and the time of reporting. Besides, the month of June is excluded from the analysis as there are generally no matches during that time of the year. In addition to the time and place (municipality level) of the crime, the date include the crime type code, the victim's age and gender, information on how the attack was carried out (attempt/completed act, usage of a firearm, lone operator/crime was committed by a group) and information on the relationship between victim and suspect.\footnote{The relationship between victim and suspect is retrieved in two ways. On the one hand, formal relationships are recorded (such as types of kinship or acquaintance). On the other hand, relationships are defined in spatial-social terms (for instance living in the same household, or being in an educational or care relationship).} Roughly 40 percent of victims are female, the average age is 32 years, and 40 percent of the victims had no prior relationship with the suspect.
% (up to the hour) 




The micro-data is aggregated to the municipality-day level and the main outcome, the assault rate, is defined as the number of assaults per million population. Assaults are defined as actions involving physical violence. For that purpose, I use the crime type code `simple willful bodily harm' (\textit{Vorsätzliche einfache Körperverletzung}, § 223 StGB). There are roughly 120 types of criminal offenses (recorded in 6 digit codes), with the vast majority of cases classified by only a handful of codes.\footnote{The top 10 of the most prevalent crime keys account for more than 90 percent of the cases.} Figure \ref{fig_soc_ext:offense_types_distribution_2014} in the Appendix shows the distribution of cases per crime key for the twenty most common offense types in 2014. `Simple willful bodily harm' is by far the most common offense, contributing to about 45 percent of all cases. Due to its prevalence and its association with aggression, I choose this offense type as the main outcome variable in the analysis.

Figure \ref{fig_soc_ext:assault_time_distribution_2014} shows the distribution of assaults over time. Panel A displays the variation of assaults per hour of the day.\footnote{Roughly 15 percent of the observations do not contain hourly information. This has no consequences for the main analysis, as I examine daily variation in the assault rate.} The number of assaults increases during the day and peaks around midnight. To assign the cases that occur in the early morning hours to the day on which they originate, I define a day as beginning at 6:00\textsc{am} and ending at 5:59\textsc{am}. Panel B shows the distribution of assaults by day of the week. There are relatively more assaults on Fridays and Saturdays, whereas the other days exhibit slightly smaller assault rates. Panel C shows that the number of assaults has a strong seasonal pattern, with the highest value recorded in May and the smallest in August.\footnote{Panel C shows the monthly number of assaults while adjusting for the number of days per month.} Panel D confirms this impression by plotting the daily number of assaults. New Year's Eve is a particularly impressive outlier.\footnote{Panels C and D show data for the year 2014 only.}

% XXX per day roughly around 320 per day, peak on new year's eve, high on holidays (carneval) 





\subsection{Football Data}
The data on football matches is self-collected and is obtained via web scraping from \url{www.kicker.de} and \url{www.transfermarkt.com}. All matches played in the first three leagues of the German football league system in the period from January 2011 until May 2015 are recorded. The data contains detailed match and table standings parameters, e.g. time and place of the match, number of spectators, pregame point difference, goals, penalties, cards, referee characteristics, among others. Furthermore, there is comprehensive information on the individual teams, such as team size, average age, market value, and the number of foreign players. The stadiums where the matches take place are geographically encoded. Figure \ref{fig_soc_ext:map_gem_stadiums_monitor_and_neighboring_regions} depicts a map with all 69 stadiums included in the data set.




% XXX stadiums in which not at least a full season is played are excluded from the analysis.\footnote{This affects the stadiums in Lübeck (1), Reutlingen (12), Elversberg (14), and the Air-Berlin stadium in Dusseldorf (3). The number in parenthesis indicates the number of games that were held in that venue. This excludes 0.6 percent of all matches.}


Figure \ref{fig_soc_ext:descriptives_matches_time_attendance} illustrates insights into key variables. Panel A shows the number of matches per day of the week and league. The vast majority of matches takes place between Friday and Sunday. Games of the lower leagues occasionally also take place during the week. Such games are held only in the evenings. In contrast, matches on weekends usually take place in the afternoon. The inclusion of day-of-week fixed effects in the baseline specification helps to account for the higher share of games played on weekends, which are associated with higher levels of criminal behavior. Spectator number vary substantially across the three professional leagues, as depicted in Panel C. The \textit{Bundesliga} attracts the most spectators with an average of 44,000 viewers per game, followed by the second league with an average of 17,000 fans per match, and the lowest league attracts slightly less than 6,000 fans per game on average.


\afterpage{
% MAP : (I) WEATHER monitors & (II) NEIGHBOUR regions
\newgeometry{left=1cm,right=1cm,top=3cm,bottom=3cm} 
\begin{landscape}
	\vspace*{\fill}
	\begin{figure}[ht]\centering
		\caption{The stadiums with the closest weather monitors and neighboring regions}\label{fig_soc_ext:map_gem_stadiums_monitor_and_neighboring_regions}

		\begin{subfigure}[h]{0.4\linewidth}\centering\caption{The closest weather monitors}
			\includegraphics[width=\linewidth]{maps/soc_ext_2010_2015_Gemeinde_stadiums_nearest_monitor_100.png}
		\end{subfigure}
		\begin{subfigure}[h]{0.4\linewidth}\centering\caption{Spatial displacement - neighboring regions}
			\includegraphics[width=\linewidth]{maps/soc_ext_2010_2015_map_stadiums_neighbor_regions_zoom_100.png}
		\end{subfigure}
		\scriptsize
		\begin{minipage}{0.95\linewidth}
			\scriptsize{\emph{Notes:} The map on the left shows the stadiums used in the analysis over the seasons 2010/11 until 2014/15 (red dots) and their closest weather monitors (blue dots). The orange lines indicate how the weather monitors are assigned to the stadiums. The map on the right shows the regions that are used in the analysis for spatial displacement effects. The neighboring municipalities are chosen to be in the sample for estimating spatial displacement effects if they have a common border with a region that contains a stadium. The red dots are the stadiums, the black outlines indicate federal state boundaries.\newline \emph{Source:} Own representation with data from the Federal Institute for Research on Building, Urban Affairs and Spatial Development (BBSR).}
		\end{minipage}
	\end{figure}
	\vspace*{\fill}\clearpage
\end{landscape}
\restoregeometry
% Descriptives about football matches
\begin{figure}[H]\centering
	\includegraphics[width=0.9 \linewidth]{descriptive/soc_ect_desc_soccer_macthes_attendance.pdf}	
	\begin{minipage}{0.9\linewidth}
		\caption{Football matches}\label{fig_soc_ext:descriptives_matches_time_attendance}
		\scriptsize{\emph{Notes:} The figures show key aspects of football games in the data set. Panel A shows how the number of matches vary over the course of a week, Panel B plots the distribution of matches over the course of a day, and Panel C shows kernel densities for the number of spectators (in thousand) across the three leagues.}
	\end{minipage}
\end{figure}
}


When investigating channels of how football games may affect assaults, I exploit betting odds obtained from \url{www.oddsportal.com} via web scraping. The betting odds give an idea of pregame expectations. I translate the odds of the three game outcomes to probabilities which are the inverse of the betting odds. The probabilities serve as suitable predictors for game outcomes, as shown in Appendix Figure \ref{fig_soc_ext:pregame_probability_spread_outcomes}.\footnote{Panel A of Figure \ref{fig_soc_ext:pregame_probability_spread_outcomes} shows the close relationship between the realized score differential and the probability spread. Panels B and C demonstrate that the probability of winning increases the higher the probability spread.}



\subsection{Weather Data}
The weather data is derived from Germany's National Meteorological Service (\textit{Deutscher Wetterdienst}). In order to construct the weather control variables, I use those weather monitors which measure the relevant weather variables in the sample period.\footnote{I use daily averages of the following weather variables: daily average, minimum and maximum air temperature, minimum ground temperature, vapor pressure, air pressure, cloud cover, air humidity, precipitation, hours of sunshine, snow depth, and wind velocity.} From this set of monitors, I choose the weather monitor with the closest proximity to a stadium. The assigned monitor-stadium pairs can be found in Figure \ref{fig_soc_ext:map_gem_stadiums_monitor_and_neighboring_regions}. There is a high quality of the matches between weather monitors and stadiums, as the average distance between stadiums and monitors is 15 kilometers.\footnote{The distribution of the distance of the monitor-stadium pairs can be found in Appendix Figure \ref{fig_soc_ext:hist_distance_stadium_monitors}.} Few of the weather variables have missing data, which are filled in by propagating forward from the last valid observation to the next valid observation (i.e. `forward fill').\footnote{The weather variables with missing data are (with the share of missing data in parenthesis): Cloud cover (<1.2 percent) and snow depth (<0.6 percent).}



\subsection{Holidays}
In order to capture any variation in the crime rate between ordinary and special days, I add controls for public and school holidays, which may differ at the state level.\footnote{The data on school holidays comes from `The Standing Conference of the Ministers of Education and Cultural Affairs of the Länder in the Federal Republic of Germany' (\textit{Kultusminister Konferenz}). The data on public holidays is collected from \href{https://www.schulferien.org/deutschland/feiertage/}{\tt https://www.schulferien.org/deutschland/feiertage/}.} Furthermore, I add a dummy variable for peculiar days (New Year's Eve and the days surrounding Carnival), which are not holidays, yet certainly shift the crime rate.



\subsection{Regional Database}
The Federal Statistical Office and the statistical offices of the Länder provide a database of detailed statistics by various subject areas at a very granular spatial level. Thus, I am able to create a panel at the municipality-year level containing comprehensive information on topics such as territory, population, labor market, election results, housing, economic sectors, and public budgets. The information is used to construct weights for the regression analysis or to determine assault rates. 







%--------------------------------------------------------------------
% IDENTIFICATION
%--------------------------------------------------------------------
\bigskip
\section{Empirical Strategy}\label{sec_soc_ext:empirical_strategy}

In order to identify the causal effect of football matches on criminal behavior, I exploit within-region variation over time. To be precise, I compare the regional assault rate on a game day to the expected assault level conditional on the day of the week, month, and year, while additionally accounting for other possible confounding variation due to weather and holidays. In other words, the counterfactual regional assault rate (what would be expected on a game day in absence of the game), e.g. a Saturday in April 2012, is obtained by using the regional assault rate on other Saturdays in April 2012 with no games scheduled. %In other words, the regional assault rate on a given date without a game (say a Saturday in April 2012) serves as counterfactual for what would be expected on a game day (on a Saturday in April 2012) in absence of the game. 


The identification strategy is based on a generalized difference-in-differences approach to study the impact of football matches on violent behavior. Let Assaults$_{rdmy}$ denote the assault rate in region $r$, on day-of-the-week $d$, in month $m$ and year $y$. The assault rate is defined as the number of assaults per million population and is given by:
\begin{align}
\text{Assaults}_{rdmy} = \alpha + \beta\ (\text{Gameday}_{rdmy}) + \vartheta_r + \underbrace{\gamma_d + \eta_m + \theta_y}_{\text{date}_{dmy}} + \lambda \mathbf{X}_{rdmy} + \varepsilon_{rdmy}
\label{eq_soc_ext:model}.
\end{align}
$\text{Gameday}_{rdmy}$ is a binary variable that equals one when there is a home game, and zero otherwise. Region fixed effects $\vartheta_r$ capture time-invariant differences between regions and ensure that the identification is driven by within instead of between region variation over time. The vector $\text{date}_{dmy}$ contains fixed effects for the day-of-the-week ($\gamma_d$), month ($\eta_m$), and year ($\theta_y$). This way, the model flexibly controls for day-of-week specific heterogeneity, seasonal effects, and long-run time trends. I expand the baseline model by adding interactions of region fixed effects with all elements of $\text{date}_{dmy}$, i.e. region-by-day-of-week fixed effects, region-by-month fixed effects, and region-by-year fixed effects. The interactions account for systemic changes in the degree of violent behavior over the year for each region. The vector $\mathbf{X}_{rdmy}$ includes school and public holidays and weather controls.\footnote{Public holiday controls include binary variables (at the level of the Federal States) for All Saints' Day, Ascension Day, Assumption Day, Christmas, Corpus Christi, Epiphany, Easter, German Unity Day, Good Friday, Labor Day, New Year's Day, Penance Day, Pentecost, and Reformation Day. Moreover, it contains dummy variables for Carnival and New Year's Eve.\newline Weather controls (at the regional level) include average air temperature, maximum air temperature, minimum air temperature, minimum ground temperature, steam pressure, cloud cover, air pressure, humidity, average precipitation, hours of sunshine, snow depth, and wind speed.} I compute two-way cluster-robust standard errors to capture arbitrary correlation at the region-year and year-month levels. Observations are weighted with the population figures from the Federal Statistical Office.


% displacement effects - temporal & spatial
The implicit assumption for interpreting the parameter of interest $\beta$ as the causal effect of a home game on violent behavior is that the location and the time of a football match are orthogonal to the number of assaults, conditional on the covariates. However, displacement effects may pose a threat to identification. On the one hand, this refers to spatial displacement effects, which may occur when (violence-prone) people from distant regions visit a game. On the other hand, this includes temporal displacement effects, which happen when assaults are shifted from adjacent days to game days. In both cases, the parameter would overestimate the impact of a football match on violent behavior as the offense would have been committed regardless, but at a different time or place. To rule out the possibility that displacement effects compromise the validity of the identification strategy, I investigate the effect of football games on neighboring regions and on days adjacent to game days in section \ref{sec_soc_ext: threats}. I find that the main results, namely an increase in violent behavior in regions where football games take place, are not neutralized by a decrease in the number of assaults in surrounding regions or on days adjacent to game days.


% control group problem - away games
Given the design of the empirical approach, there could be another potential threat to the validity of the identification strategy. By focusing on home games in the main analysis, the counterfactuals may be biased downwards as days with away games are part of the control group. This control group problem may be due to violent fan groups traveling with their team to away games, potentially leading to a decline in the assault rate in the home region. To address this concern, I perform the analysis again, differentiating between home and away games. When considering distinct effects for home and away games, I find that away games do not significantly affect the assault rate in the home region. 









%--------------------------------------------------------------------
% RESULTS
%--------------------------------------------------------------------
\bigskip
\section{Results}\label{sec_soc_ext:results}



% MAIN RESULTS
%--------------------------------------------------------------------
\subsection{Main Results}

% fig: intuitive representation: assrate across DOW
\begin{figure}[H]\centering
	\caption{The average assault rate on gamedays and days where no game takes place}\label{fig_soc_ext:assault_rate_across_dows}
	\includegraphics[width=0.9 \linewidth]{descriptive/soc_ext_descr_assaultrate_gd_nongd_hor.pdf}
	\begin{minipage}{0.95\linewidth}
		\scriptsize{\emph{Notes:} The figure shows the daily average number of assaults per million population for regions that host games of a soccer team from the first three leagues of the German football league system. The daily rates are shown for weeks in which a game is played and for weeks in which no game takes place.}
	\end{minipage}
\end{figure}

% descriptive figure, ass_rate across DOW
Before presenting the regression results, Figure \ref{fig_soc_ext:assault_rate_across_dows} gives an intuitive preview of the main findings. Using the same data as in the main analysis, it shows the assault rate across the days of the week.\footnote{Appendix Figure \ref{fig_soc_ext:assaults_across_dows} shows the same scheme using the raw number of assaults across days of the week for days with and without games.} The daily rates are presented for days with and without home games. The average daily assault rate is higher for weeks when a home game is played than when no game is played. The difference in means is statistically significant for all days except Tuesdays and Wednesdays.\footnote{The average assault rate on Thursdays with games is very high compared to Thursdays without games. This may be the result of very few games take place on a Thursday. In my sample, only 18 (0.4 percent) of all games take place on a Thursday.} The empirical model exploits the variation in the number of assaults across the days of the week, and in particular how the pattern varies between weeks with and without games.


% main regression table: include fixed effects
\begin{table}[ht] \centering 
	\begin{threeparttable} \centering \caption{Effects on \textbf{Assault rate}}\label{tab_soc_ext:reg_fe_assrate}
		{\def\sym#1{\ifmmode^{#1}\else\(^{#1}\)\fi} 
			\begin{tabular}{l*{6}{c}}
				\toprule 
				&\multicolumn{1}{c}{(1)}&\multicolumn{1}{c}{(2)}&\multicolumn{1}{c}{(3)}&\multicolumn{1}{c}{(4)}\\
				\midrule
				Game day            &       2.740\sym{***}&       2.766\sym{***}&       2.813\sym{***}&       2.677\sym{***}\\
                    &     (0.319)         &     (0.312)         &     (0.313)         &     (0.284)         \\
\midrule Effect size [\%]&       22.00         &       22.21         &       22.59         &       21.50         \\
Observations        &      88,028         &      88,028         &      88,028         &      88,028         \\
Region FE           &         yes         &         yes         &         yes         &         yes         \\
Time Fe             &         yes         &         yes         &         yes         &         yes         \\
Weather Controls    &           -         &         yes         &         yes         &         yes         \\
Holiday FE          &           -         &           -         &         yes         &         yes         \\
Interact FE         &           -         &           -         &           -         &         yes         \\
 
				\bottomrule 
		\end{tabular}}
		\begin{tablenotes} 
			\item \scriptsize \emph{Notes:} Estimates are based on the model shown in equation \ref{eq_soc_ext:model}. The specifications use daily data (excluding June) spanning the time window 2011-2015 for regions that host games of a soccer team from the first three leagues of the German football league system. The outcome variable is defined as the number of assaults per million population. Population-weighted coefficients show the change in the outcome variable due to a home game. Days are defined to run from 6:00 AM until 5:59 AM the following day to accommodate the fact that offenses committed in the early hours have their origin in the preceding day. The effect size corresponds to the percent change of the assault rate due to a soccer game in relation to the mean when no game takes place. Control variables shown as \textit{Date FE} include dummies for day-of-week, month, and year. \textit{Weather controls} include air temperature (average, maximum, and minimum), minimum ground temperature, vapor pressure, air pressure, cloud cover, air humidity, precipitation, sun shine duration, snow depth and wind velocity. \textit{Holiday FE} is a dummy variable for public and school holidays, as well as for other peculiar days. Control variables shown as \textit{Interact FE} consist of interactions of region dummies with all elements of the date fixed effects. Two-way clustered standard errors at region-year and year-month level are reported in parentheses. \newline Significance levels: * p < 0.10, ** p < 0.05, *** p < 0.01.
		\end{tablenotes} 
	\end{threeparttable} 
\end{table}

% main result table with more controls
Table \ref{tab_soc_ext:reg_fe_assrate} reports estimates corresponding to equation \ref{eq_soc_ext:model} when continuously adding more controls. The dependent variable is the assault rate, which is defined as the number of offenses per million population. In column 1, I include region, day-of-week, month, and year fixed effects. In column 2 weather controls are added. Holidays fixed effects are included in column 3. In column 4, I control for region-specific date fixed effects. Note that although the estimates vary marginally across columns, I use throughout the paper the model presented in column 4 for the analyses that follow. The estimate from the preferred specification in column 4 suggests that a home game increases the assault rate by an average of 2.677 assaults per million population. This corresponds to an increase of 21.5 percent compared to the level without games.




% table effect heterogeneity: victim and crime chars
% This file combines manually the coefficients from the tables of victim and crime characteristics


\begin{table}[t] \centering 
	\begin{threeparttable} \centering \caption{Effect heterogeneity by victim and crime characteristics}
		\label{tab_soc_ext:reg_fe_assrate_victim_crime_chars}
		{\def\sym#1{\ifmmode^{#1}\else\(^{#1}\)\fi} 
			\begin{tabular}{l*{6}{c}}
				\toprule 
				&\multicolumn{1}{c}{(1)}&\multicolumn{1}{c}{(2)}&\multicolumn{1}{c}{(3)}&\multicolumn{1}{c}{(4)}&\multicolumn{1}{c}{(5)}&\multicolumn{1}{c}{(6)}\\
				\midrule

				\\
				\multicolumn{6}{l}{\textit{\textbf{Panel A: Victim characteristics}}} \\\\
					& \multicolumn{2}{c}{Gender} & \multicolumn{3}{c}{Victim-suspect-relationship} & Occupation \\
					\cmidrule(lr){2-3} \cmidrule(lr){4-6} \cmidrule(lr){7-7}
				&  women & men & strangers$^a$ & \clb{c}{prior$^a$\\relation} & domestic$^b$ & police \\
				\midrule			
				Game day           		 &     0.245\sym{***}&       2.432\sym{***}&       1.939\sym{***}&       0.738\sym{***}&       0.096\sym{**} &       0.434\sym{***}\\
				                    	 &   (0.090)         &     (0.232)         &     (0.207)         &     (0.116)         &     (0.045)         &     (0.078)         \\
				Effect size [\%]&      4.99         &       32.27         &       40.43         &        9.64         &        5.40         &       96.98         \\
				Observations        	 &    88,028         &      88,028         &      88,028         &      88,028         &      88,028         &      88,028         \\
				
				\\\\\\
				\multicolumn{6}{l}{\textit{\textbf{Panel B: Crime characteristics}}} \\\\
				& \multicolumn{4}{c}{Timing$^c$} & \multicolumn{2}{c}{Assault execution} \\
				\cmidrule(lr){2-5} \cmidrule(lr){6-7}
				& spring & summer & fall & winter & attempt & \clb{c}{completed\\act} \\
				\midrule
				Game day            	 &     2.817\sym{***}&       1.482\sym{**} &       2.478\sym{***}&       2.849\sym{***}&       0.367\sym{***}&       2.310\sym{***}\\
										 &   (0.404)         &     (0.484)         &     (0.372)         &     (0.535)         &     (0.060)         &     (0.238)         \\
				Effect size [\%]&     21.00         &       13.40         &       20.71         &       22.51         &       56.21         &       19.58         \\
				Observations        	 &    27,140         &      14,632         &      21,476         &      24,780         &      88,028         &      88,028         \\
				
				
%				Region FE           &              \checkmark         &         \checkmark         &         \checkmark         &         \checkmark         &         \checkmark         &         \checkmark         \\
%				Date Fe             &              \checkmark         &         \checkmark         &         \checkmark         &         \checkmark         &         \checkmark         &         \checkmark         \\
%				Weather Controls    &              \checkmark         &         \checkmark         &         \checkmark         &         \checkmark         &         \checkmark         &         \checkmark         \\
%				Holiday FE          &              \checkmark         &         \checkmark         &         \checkmark         &         \checkmark         &         \checkmark         &         \checkmark         \\
%				Interact FE         &              \checkmark         &         \checkmark         &         \checkmark         &         \checkmark         &         \checkmark         &         \checkmark         \\
				\bottomrule 
		\end{tabular}}
		\begin{tablenotes} 
			\item \scriptsize \emph{Notes:} The estimates are based on the model shown in equation \ref{eq_soc_ext:model} and use the same outcome and controls as column 4 of Table \ref{tab_soc_ext:reg_fe_assrate} (including region and date fixed effects, their interactions as well as holiday and weather controls). See Table \ref{tab_soc_ext:reg_fe_assrate} for additional details. Two-way clustered standard errors at region-year and year-month level are reported in parentheses. \newline Significance levels: * p < 0.10, ** p < 0.05, *** p < 0.01.\newline \hspace*{15pt} $^a$: covers formal relationships (e.g. types of kinship or acquaintance). \newline \hspace*{15pt} $^b$: covers spatial-social relationships (whether victim and suspect live in the same household). \newline \hspace*{15pt} $^c$: spring (Mar-May), summer (Jul+Aug), fall (Sep-Nov), winter (Dec-Feb).
		\end{tablenotes} 
	\end{threeparttable} 
\end{table}




% original notes
% Estimates are based on the model shown in equation \ref{eq_soc_ext:model}. The specifications use daily data (excluding June) spanning the time window 2011-2015 for regions that host games of a soccer team from the first three leagues of the German football league system. The outcome variable is defined as the number of assaults per million population. Population-weighted coefficients show the change in the outcome variable due to a home game. Days are defined to run from 6:00 AM until 5:59 AM the following day to accommodate the fact that offenses committed in the early hours have their origin in the preceding day. The effect size corresponds to the percent change of the assault rate due to a soccer game in relation to the mean when no game takes place. Control variables shown as \textit{Date FE} include dummies for day-of-week, month, and year. \textit{Weather controls} include air temperature (average, maximum, and minimum), minimum ground temperature, vapor pressure, air pressure, cloud cover, air humidity, precipitation, sun shine duration, snow depth and wind velocity. \textit{Holiday FE} is a dummy variable for public and school holidays, as well as for other peculiar days. Control variables shown as \textit{Interact FE} consist of interactions of region dummies with all elements of the date fixed effects. Two-way clustered standard errors at region-year and year-month level are reported in parentheses. \newline Significance levels: * p < 0.10, ** p < 0.05, *** p < 0.01.\newline \hspace*{15pt} $^a$: covers formal relationships (e.g. types of kin or acquaintances). \newline \hspace*{15pt} $^b$: covers spatial-social relationships. In this instance, whether victim and suspect live in the same household.\newline \hspace*{15pt} $^c$: spring (March - May), summer (July and August), fall (September - November), winter (December - February).

% victim characteristics
Having established that football games lead to more assaults, I now shed light on the victims of these additional offenses. Panel A of Table \ref{tab_soc_ext:reg_fe_assrate_victim_crime_chars} shows the effect of football games on the assault rate by victim characteristics. First, columns 1 and 2 show the effect heterogeneity by gender. Although the estimates for women and men are statistically different from zero, the vast majority of additional victims are male. Male victimization rates increase on average by 2.432 offenses per million population. The increased assault rate for males accounts for more than 90 percent of the effect found for the entire sample. Figure \ref{fig_soc_ext:fe_age_profile} shows the age profile of the impact of football games on the assault rate for each gender. For women, the point estimates are small in magnitude and not significantly different from zero. Women aged 40-49 are the only exception. In contrast, the effects for adult men are throughout significant. The largest effect for males is found in the 18-29 age group and decreases thereafter. Second, I explore effect heterogeneity according to the relationship between victim and suspect. Columns 3 and 4 distinguish the relationships from a formal perspective, such as kinship or acquaintance. Although both estimates are positive and statistically significant, the majority of additional assaults involves victims with no prior connection to the suspect. The victimization rate of strangers to the suspect increases by on average 1.939 assaults per million population. This implies that almost three out of four additional cases involve this type of victim-suspect pairing. Column 5 considers spatial-social relationships, namely whether victim and suspect live in the same household. A football game increases the number of domestic assaults by on average 0.096 assaults per million population. This implies that only few (13 percent) victims who know the suspects also live in the same household with them. Third, the effect of the victim's occupation is investigated. Column 6 shows the impact of football games on violent behavior directed at police officers on duty. The victimization rate of police officers increases on average by 0.434 assaults per million population. Thus, almost 16 percent of the additional assaults resulting from football games are attributed to attacks on police officers. In short, the additional victims are young males with no prior relationship to the suspect. Hence, the victims and the people who usually gather in and around football stadiums are similar in demographic terms \citep{pwc2016}.

% age profile of effect, ba gender
\begin{figure}[H]\centering
	\caption{The age profile of the impact of soccer matches on the assault rate}\label{fig_soc_ext:fe_age_profile}
	% \begin{subfigure}[h]{0.31\linewidth}\centering\caption{Total}
	% 	\includegraphics[width=\linewidth]{regression/fe_agebrackets_t.pdf}
	% \end{subfigure}
	\begin{subfigure}[h]{0.48\linewidth}\centering\caption{Women}
		\includegraphics[width=\linewidth]{regression/fe_agebrackets_f.pdf}
	\end{subfigure}
	\begin{subfigure}[h]{0.48\linewidth}\centering\caption{Men}
		\includegraphics[width=\linewidth]{regression/fe_agebrackets_m.pdf}
	\end{subfigure}
	\scriptsize
	\begin{minipage}{\linewidth}
		\emph{Notes:} The figure shows estimates and 95\% confidence intervals across age brackets and by gender. To estimate the effect, I used the model shown in equation \ref{eq_soc_ext:model} in its richest specification with all covariates and population weights. The outcome variable is defined as the number of assaults per million population.
	\end{minipage}
\end{figure}

% crime characteristics
In the next step, I consider effect heterogeneity by crime characteristics. Columns 1-4 in Panel B of Table \ref{tab_soc_ext:reg_fe_assrate_victim_crime_chars} show the heterogeneity of the effect by the time of the offense. The impact of a football game on the assault rate is smallest in summer months, with 1.482 assaults per million population, representing an increase of 13.4 percent. The football season starts in the summer months (July and August). At this point, the results of the games are not yet so important. The estimates are higher (but not significantly different from each other) at other times of the season. Columns 5 and 6 consider effect heterogeneity by assault execution (attempt/completed offense). I find that the effects of football games on the assault rate are mostly driven by completed offenses with on average 2.310 additional assaults per million population. The share of attempted assaults also increases, but at a smaller rate of 0.367 additional attempted assaults per million population on average.




% Threats to validity
%--------------------------------------------------------------------
\subsection{Potential Threats to Identification and Validity of the Design}\label{sec_soc_ext: threats}
In this section, I consider the possibility that my results do not reflect additional assaults due to football games, but merely shifts in offenses. Furthermore, I test the sensitivity of the main results by additionally including away games in the model.

% Intro why we see displacement
Previously, I have presented evidence that football games increase the assault rate. However, it is possible that I only capture an effect that shifts offenses. For example, the increase in violent behavior may be offset by a decline in assaults in other areas or at different times \citep{lindo2018college}. In other words, the assault would have been committed regardless, but at a different time or location. One explanation may be different population flows around days on which games take place.

% table SPATIAL & TEMPORAL DISPLACEMENT 
\begin{table}[ht] \centering 
	\begin{threeparttable} \centering \caption{Displacement effects}\label{tab_soc_ext:reg_fe_assrate_displacement}
		{\def\sym#1{\ifmmode^{#1}\else\(^{#1}\)\fi} 
			\begin{tabular}{l*{6}{c}}
				\toprule 
				&\multicolumn{1}{c}{(1)}&\multicolumn{1}{c}{(2)}&\multicolumn{1}{c}{(3)}&\multicolumn{1}{c}{(4)}\\
				\midrule
				
% THIS TABLE WAS GENERATED BY ADDING ROWS OF 
%	1) assrate_neighbor_regions_fixed_effects
%	2) assrate_fixed_effects_leads_lags
\\
\textit{\textbf{Panel A: Spatial displacement}}\\
Game day            &       0.144         &       0.153         &       0.170         &       0.210         \\
                    &     (0.146)         &     (0.146)         &     (0.147)         &     (0.154)         \\
Effect size [\%]&        2.58         &        2.74         &        3.04         &        3.75         \\
Observations        &     960,848         &     960,848         &     960,848         &     960,848         \\

\\\\
\textit{\textbf{Panel B: Temporal displacement}}\\
Game day            &       2.777\sym{***}&       2.810\sym{***}&       2.857\sym{***}&       2.770\sym{***}\\
                    &     (0.346)         &     (0.338)         &     (0.336)         &     (0.300)         \\
Day after game      &      -0.105         &      -0.076         &      -0.065         &       0.279         \\
                    &     (0.189)         &     (0.190)         &     (0.192)         &     (0.182)         \\
Day before game     &       0.287         &       0.318         &       0.339         &       0.352\sym{*}  \\
                    &     (0.241)         &     (0.238)         &     (0.229)         &     (0.206)         \\
Effect size [\%]&       22.34         &       22.60         &       22.98         &       22.28         \\
Observations        &      87,438         &      87,438         &      87,438         &      87,438         \\
 \\\\
 Region FE           &         \checkmark         &         \checkmark         &         \checkmark         &         \checkmark         \\
 Date FE             &         \checkmark         &         \checkmark         &         \checkmark         &         \checkmark         \\
 Weather Controls    &           -         &         \checkmark         &         \checkmark         &         \checkmark         \\
 Holiday FE          &           -         &           -         &         \checkmark         &         \checkmark         \\
 Interact FE         &           -         &           -         &           -         &         \checkmark         \\
 
				\bottomrule 
		\end{tabular}}
		\begin{tablenotes} 
			\item \scriptsize \emph{Notes:} Estimates are based on the model shown in equation \ref{eq_soc_ext:model}. Panel A contains specifications that use daily data (excluding June) spanning the time window 2011-2015 for regions that share a common boundary with a distract that contains a stadium of a soccer team from the first three leagues of the German football league system. Panel B shows specifications that use daily data (excluding June) spanning the time window 2011-2015 for regions that host games of a soccer team from the first three leagues of the German football league system.	The outcome variable is defined as the number of assaults per million population. Population-weighted coefficients show the change in the outcome variable due to a home game. Days are defined to run from 6:00 AM until 5:59 AM the following day to accommodate the fact that offenses committed in the early hours have their origin in the preceding day. The effect size corresponds to the percent change of the assault rate due to a soccer game in relation to the mean when no game takes place. Control variables shown as \textit{Date FE} include dummies for day-of-week, month, and year. \textit{Weather controls} include air temperature (average, maximum, and minimum), minimum ground temperature, vapor pressure, air pressure, cloud cover, air humidity, precipitation, sun shine duration, snow depth and wind velocity. \textit{Holiday FE} is a dummy variable for public and school holidays, as well as for other peculiar days. Control variables shown as \textit{Interact FE} consist of interactions of region dummies with all elements of the date fixed effects. Two-way clustered standard errors at region-year and year-month level are reported in parentheses. \newline Significance levels: * p < 0.10, ** p < 0.05, *** p < 0.01.
		\end{tablenotes} 
	\end{threeparttable} 
\end{table}

% spatial displacement
In order to estimate \textit{spatial displacement effects}, I investigate the impact of football matches on neighboring regions. A neighboring region is a municipality that shares a border with a region in which a stadium is located.\footnote{If two municipalities share a border and each of the regions contains a stadium, both regions will not serve as neighbor regions and they are dismissed from the set of spatial spillover candidates.} Figure \ref{fig_soc_ext:map_gem_stadiums_monitor_and_neighboring_regions} shows a map of the selected regions. The sample of neighboring regions exhibits a considerably higher number of observations. This is owed to the fact that a region with a stadium has on average slightly more than 11 neighboring municipalities. Panel A of Table \ref{tab_soc_ext:reg_fe_assrate_displacement} shows the estimates of the impact of a home game on these neighboring regions. In comparison to the baseline effects, the spatial spillover coefficients are small and not significantly different from zero. Consequently, the results do not suggest offsetting spatial spillover effects.

% temporal displacement
In the next step, \textit{temporal displacement effects} are considered. To capture these effects, I include a one-day lead and lag of the game day indicator. Panel B of Table \ref{tab_soc_ext:reg_fe_assrate_displacement} contains the estimates when including the temporal spillover components in the baseline model. The estimates of the game day itself are not significantly different from the baseline model. Almost all of the coefficients for the day before and after the game are small in magnitude and not statistically significant. The only exception is the estimate for the day before the game in column 4, which implies that a football game leads to an average increase of 0.352 assaults per million population on the day before a game. The positive coefficient, however, does not suggest temporal displacement effects, which would require a negative estimate. Rather, it indicates additional assaults due to activity on the day preceding important games. 

% table HOME & AWAY games 
\begin{table}[t] \centering 
	\begin{threeparttable} \centering \caption{Effect of soccer games on the assault rate, distinction of home and away games}\label{tab_soc_ext:reg_fe_assrate_home_away}
		{\def\sym#1{\ifmmode^{#1}\else\(^{#1}\)\fi} 
			\begin{tabular}{l*{2}{c}}
				\toprule 
				&\multicolumn{1}{c}{(1)}&\multicolumn{1}{c}{(2)}\\
				\midrule
				&\clb{c}{baseline\\w/o L3} & \clb{c}{distinction\\home/away$^1$}\\
				\midrule
				Game day            &       2.859\sym{***}&                     \\
                    &     (0.324)         &                     \\
Home game day       &                     &       2.910\sym{***}\\
                    &                     &     (0.335)         \\
Away game day       &                     &       0.325         \\
                    &                     &     (0.213)         \\
\midrule Effect size [\%]&       21.22         &       21.97         \\
Observations        &      61,172         &      61,172         \\
Region FE           &         \checkmark         &         \checkmark         \\
Date Fe             &         \checkmark         &         \checkmark         \\
Weather Controls    &         \checkmark         &         \checkmark         \\
Holiday FE          &         \checkmark         &         \checkmark         \\
Interact FE         &         \checkmark         &         \checkmark         \\
 
				\bottomrule 
		\end{tabular}}
		\begin{tablenotes} 
			\item \scriptsize \emph{Notes:} The specifications use daily data (excluding June) spanning the time window 2011-2015 for regions that host games of a soccer team from the first two leagues of the German football league system. The estimates are based on the model shown in equation \ref{eq_soc_ext:model} and use the same set of controls as column 4 of Table \ref{tab_soc_ext:reg_fe_assrate} (including region and date fixed effects, their interactions as well as holiday and weather controls). See Table \ref{tab_soc_ext:reg_fe_assrate} for additional details. Two-way clustered standard errors at region-year and year-month level are reported in parentheses. \newline Significance levels: * p < 0.10, ** p < 0.05, *** p < 0.01.\newline
			\hspace*{15 pt}$^1$: Effect size corresponds to the coefficient of Home game day.
		\end{tablenotes} 
	\end{threeparttable} 
\end{table}

% The specifications use daily data (excluding June) spanning the time window 2011-2015 for regions that host games of a soccer team from the first two leagues of the German football league system. The outcome variable is defined as the number of assaults per million population. Population-weighted coefficients show the change in the outcome variable due to a home game. Days are defined to run from 6:00 AM until 5:59 AM the following day to accommodate the fact that offenses committed in the early hours have their origin in the preceding day. The effect size corresponds to the percent change of the assault rate due to a soccer game in relation to the mean when no game takes place. Control variables shown as \textit{Date FE} include dummies for day-of-week, month, and year. \textit{Weather controls} include air temperature (average, maximum, and minimum), minimum ground temperature, vapor pressure, air pressure, cloud cover, air humidity, precipitation, sun shine duration, snow depth and wind velocity. \textit{Holiday FE} is a dummy variable for public and school holidays, as well as for other peculiar days. Control variables shown as \textit{Interact FE} consist of interactions of region dummies with all elements of the date fixed effects. Two-way clustered standard errors at region-year and year-month level are reported in parentheses. \newline Significance levels: * p < 0.10, ** p < 0.05, *** p < 0.01.\newline \hspace*{15 pt}$^1$: Effect size corresponds to the coefficient of Home game day.

% Home and Away games
As discussed above, the baseline model considers the effect of football matches, but only for home games. This restriction may compromise the validity of the design. When away matches are not accounted for, they end up in the control group. The control observations might be biased downwards if the most devoted (and possibly violent) fans leave their home municipality to accompany their local team to an away game. The resulting decrease in the assault rate at home due to the absence of local agitators imply that days with away matches can no longer function as control units. To address this concern, I investigate the \textit{effect of home and away matches separately}. To analyze the effect of away matches, the design of my data set must be modified. In the baseline version, the football data is merged with other datasets at the match level (the region ID of the home team serves as the identifier). In this case, I use the football data at the table standings level. In other words, both the home and away teams are matched with a region. This approach leads to ambiguity regarding the treatment status of individual regions.\footnote{When only considering home games, the treatment status is not a problem. This is due to the fact that local authorities do not allow two home games on the same day.} For instance, the treatment status of regions with more than one team is ambiguous when there is a home and an away match on the same day. To alleviate this concern, I exclude the third league from the sample and focus exclusively on the first two leagues.\footnote{Some ambiguities remain, but they are solved as follows: 12 percent of the matches still include a duplication of two teams per region playing on the same day, either one home and one away game, or two away games. In the latter case, the status of the region is defined as `away'. In the former case, it is defined as `home'.} This approach helps considerably to clarify the treatment status of a region. Table \ref{tab_soc_ext:reg_fe_assrate_home_away} shows the results when home and away matches are examined separately. To compare the estimated effects, column 1 shows estimates retrieved from the baseline model (home matches only) when the sample is adjusted as described above. Column 2 presents the estimates that incorporate the impact of home and away games on the assault rate. The effect of a home game is sizable, leading to an average increase in the assault rate of 2.910 assaults per million population. The coefficient is not significantly different from that in column 1. A negative and significant estimate of an away game would compromise the identification strategy. However, the estimate of an away game is small in magnitude and not significantly different from zero. Thus, the results suggest that focusing exclusively on home games does not render the identification strategy invalid. 




% Channels
%--------------------------------------------------------------------
\subsection{Channels}
This section investigates potential mechanisms through which football games may cause additional assaults. I consider emotional cues and the prominence of games as potential channels.



% emotional cues
First, I consider emotional cues. This is motivated by the findings of \cite{card2011family} who demonstrate that unexpected defeats of local football teams trigger family violence. The results can be best explained with the frustration-aggression hypothesis, first proposed by \cite{dollard1939frustration}, which predicts aggressive behavior in the event of frustrating events. \cite{rees2009college} similarly show that there are more violent offenses when the local college football team suffers a defeat. For this reason, I analyze in Table \ref{tab_soc_ext:reg_fe_assrate_emotional_cues} whether visceral factors may be the reason for the additional assaults. In column 1 I investigate whether an emotionally upsetting event during a game leads to a higher assault rate. To answer this question, I create an index that equals one for games that include at least one of the following potentially troubling events: a penalty is awarded (20 percent of all games), a player receives a red card (10 percent of all games), or the referee receives an insufficient grade (15 percent of all games). The index shows that 35 percent of all games involve at least one upsetting episode as defined in the previous categories. The estimates in column 1 do not suggest that emotional cues trigger more violent behavior since the estimates for games with and without upsetting events are not significantly different. In the second and third column, I show estimates following the approach of \cite{card2011family}. I examine the impact of game outcomes relative to their pregame expectations. Pregame expectations are included in the analysis as matches with contrasting predictions may be very different from each other. By including predicted outcomes, I can estimate the effect that results from the defeat of a team that was expected to win, and vice versa. Using data from oddsportal.com, I define a game as unpredictable when the absolute probability difference between winning and loosing is smaller than 20 percentage points.\footnote{Similar results are obtained when I define different threshold values and when I deviate from the symmetry around the origin.} When the spread's value exceeds the threshold, a win or a loss of the home game is expected. Around 45 percent of the games are expected to be close, another 45 percent are expected to be won, while 10 percent of the games are expected to be lost. The significantly larger share of expected victories may be attributed to the home-advantage. In column 2, I first examine the effects of matches with distinct predicted match outcomes. The estimates do not suggest that the effect of games with different predicted outcomes vary systematically from each other. In column 3, I additionally include interactions between expected and actual game outcomes. The estimates are relatively small in magnitude and not significantly different from zero, implying that unexpected wins/losses do not cause additional assaults. Altogether, there is no evidence that emotional cues drive violent behavior in the context of professional football games in Germany.

% table EMOTIONAL CUES
\begin{table}[H] \centering 
	\begin{threeparttable} \centering \caption{Effect of emotional cues}\label{tab_soc_ext:reg_fe_assrate_emotional_cues}
		{\def\sym#1{\ifmmode^{#1}\else\(^{#1}\)\fi} 
			\begin{tabular}{l*{3}{c}}
				\toprule 
				&\multicolumn{1}{c}{(1)}&\multicolumn{1}{c}{(2)}&\multicolumn{1}{c}{(3)}\\
				&&\multicolumn{2}{c}{Card \& Dahl (2011) specification}\\
				\cmidrule(lr){3-4}
				&\clb{c}{Upset\\event index} & \clb{c}{predicted\\outcomes}& \clb{c}{predicted\\and actual\\outcomes}\\
				\midrule
				Upset event (Index) &       2.717\sym{***}&                     &                     \\
                    &     (0.406)         &                     &                     \\
No upset event (Index)&       2.655\sym{***}&                     &                     \\
                    &     (0.289)         &                     &                     \\
Expected to lose    &                     &       3.495\sym{***}&       3.376\sym{***}\\
                    &                     &     (0.722)         &     (0.836)         \\
Expected to win     &                     &       2.752\sym{***}&       2.907\sym{***}\\
                    &                     &     (0.366)         &     (0.356)         \\
Expected to be close&                     &       2.437\sym{***}&       2.369\sym{***}\\
                    &                     &     (0.321)         &     (0.383)         \\
Expected to lose and won (upset win)&                     &                     &       0.599         \\
                    &                     &                     &     (1.554)         \\
Expected to be close and lost (upset loss)&                     &                     &       0.194         \\
                    &                     &                     &     (0.610)         \\
Expected to win and lost (upset loss)&                     &                     &      -0.792         \\
                    &                     &                     &     (0.667)         \\
\midrule Observations&      88,028         &      88,028         &      88,028         \\
Region FE           &         yes         &         yes         &         yes         \\
Time Fe             &         yes         &         yes         &         yes         \\
Weather Controls    &         yes         &         yes         &         yes         \\
Holiday FE          &         yes         &         yes         &         yes         \\
Interact FE         &         yes         &         yes         &         yes         \\
 
				\bottomrule 
		\end{tabular}}
		\begin{tablenotes} 
			\item \scriptsize \emph{Notes:} Estimates are based on the model shown in equation \ref{eq_soc_ext:model}. The gameday indicator is replaced by an index that captures unsettling events. The specifications use daily data (excluding June) spanning the time window 2011-2015 for regions that host games of a soccer team from the first three leagues of the German football league system. The outcome variable is defined as the number of assaults per million population. Population-weighted coefficients show the change in the outcome variable due to a home game. Days are defined to run from 6:00 AM until 5:59 AM the following day to accommodate the fact that offenses committed in the early hours have their origin in the preceding day. Control variables shown as \textit{Date FE} include dummies for day-of-week, month, and year. \textit{Weather controls} include air temperature (average, maximum, and minimum), minimum ground temperature, vapor pressure, air pressure, cloud cover, air humidity, precipitation, sun shine duration, snow depth and wind velocity. \textit{Holiday FE} is a dummy variable for public and school holidays, as well as for other peculiar days. Control variables shown as \textit{Interact FE} consist of interactions of region dummies with all elements of the date fixed effects. Two-way clustered standard errors at region-year and year-month level are reported in parentheses. \newline Significance levels: * p < 0.10, ** p < 0.05, *** p < 0.01. \newline 	The upset event index in column 1 is defined as a dummy variable equal to one if one of the following events take place: a penalty is awarded (20\% of all games), a red card is being issued (10\% of all games), or the referee receives a non-sufficient grade (15\% of all games). In columns 2 and 3 I use data from oddsportal.com to classify games as expected to win/lose/be close.
		\end{tablenotes} 
	\end{threeparttable} 
\end{table}

% table TEAM/GAME prominence
\begin{table}[H] \centering 
	\begin{threeparttable} \centering \caption{Effect of game/team prominence}\label{tab_soc_ext:reg_fe_assrate_prominent_games}
		{\def\sym#1{\ifmmode^{#1}\else\(^{#1}\)\fi} 
			\begin{tabular}{l*{5}{c}}
				\toprule 
				&\multicolumn{1}{c}{(1)}&\multicolumn{1}{c}{(2)}&\multicolumn{1}{c}{(3)}&\multicolumn{1}{c}{(4)}&\multicolumn{1}{c}{(5)}\\
				&\multicolumn{2}{c}{Rivals/derbies} & \multicolumn{3}{c}{League}\\
				\cmidrule(lr){2-3}\cmidrule(lr){4-6}
				& rival & non-rival & league 1 & league 2 & league 3\\
				\midrule
				Game day            &       8.320\sym{***}&       2.387\sym{***}&       3.530\sym{***}&       1.878\sym{***}&       1.569\sym{***}\\
                    &     (1.746)         &     (0.275)         &     (0.460)         &     (0.318)         &     (0.350)         \\
\midrule Effect size [\%]&       66.81         &       19.17         &       28.35         &       15.08         &       12.60         \\
Observations        &      88,028         &      88,028         &      88,028         &      88,028         &      88,028         \\
%Region FE           &         \checkmark         &         \checkmark         &         \checkmark         &         \checkmark         &         \checkmark         \\
%Date Fe             &         \checkmark         &         \checkmark         &         \checkmark         &         \checkmark         &         \checkmark         \\
%Weather Controls    &         \checkmark         &         \checkmark         &         \checkmark         &         \checkmark         &         \checkmark         \\
%Holiday FE          &         \checkmark         &         \checkmark         &         \checkmark         &         \checkmark         &         \checkmark         \\
%Interact FE         &         \checkmark         &         \checkmark         &         \checkmark         &         \checkmark         &         \checkmark         \\
 
				\bottomrule 
		\end{tabular}}
		\begin{tablenotes} 
			\item \scriptsize \emph{Notes:} Estimates are based on the model shown in equation \ref{eq_soc_ext:model}. The gameday indicator is replaced by interactions with dummy variables for rival matches and league affiliation. The specifications use daily data (excluding June) spanning the time window 2011-2015 for regions that host games of a soccer team from the first three leagues of the German football league system. The outcome variable is defined as the number of assaults per million population. Population-weighted coefficients show the change in the outcome variable due to a home game. Days are defined to run from 6:00 AM until 5:59 AM the following day to accommodate the fact that offenses committed in the early hours have their origin in the preceding day. The effect size corresponds to the percent change of the assault rate due to a soccer game in relation to the mean when no game takes place. Control variables shown as \textit{Date FE} include dummies for day-of-week, month, and year. \textit{Weather controls} include air temperature (average, maximum, and minimum), minimum ground temperature, vapor pressure, air pressure, cloud cover, air humidity, precipitation, sun shine duration, snow depth and wind velocity. \textit{Holiday FE} is a dummy variable for public and school holidays, as well as for other peculiar days. Control variables shown as \textit{Interact FE} consist of interactions of region dummies with all elements of the date fixed effects. Two-way clustered standard errors at region-year and year-month level are reported in parentheses. \newline Significance levels: * p < 0.10, ** p < 0.05, *** p < 0.01.
		\end{tablenotes} 
	\end{threeparttable} 
\end{table}


% game/team prominence (rivals & league)
Second, I examine game and team prominence as potential channels. For many sports fans, the affection for their football team plays a central role \citep{wann1993sports}. Almost as important is the cultivation of animosities against rivals. Resentment and hatred is likely to lead to violent actions \citep{nassauer2011hate}. To test the hypothesis that the additional assaults are the result of hostile feelings, I compare the impact of matches played between known rival teams to regular matches. The game day indicator from equation \ref{eq_soc_ext:model} is replaced by an interaction with a dummy variable that equals one for high-rivalry matches. Local derbies (games between two competing teams that are based in regions of close geographical proximity) constitute high-rivalry matches.\footnote{Appendix Table \ref{tab_soc_ext:prominent_games_derbies} gives an overview of high-rivalry matches.} The estimates are shown in columns 1 and 2 of Table \ref{tab_soc_ext:reg_fe_assrate_prominent_games}. Games that are classified as high-rivalry matches lead to an average increase of 8.320 assaults per million population. This corresponds to more than three times the baseline effect. Although the standard errors are relatively large, considering that only 2.5 percent of the games are classified as high-rivalry matches, the effect is significantly different from that for regular matches. The higher rate of physical assaults during high-rivalry matches could also be rationalized by the social learning theory, according to which spectators mimic behavior seen on the field. For instance, the number of yellow and red cards is significantly higher in high-rivalry matches.\footnote{The number of cards in derby matches is 4.324\ (s.e.=0.174), while the number in regular matches is 3.906 (s.e.=0.029).} Finally, the mere prominence of teams/games might attract more (violent) people. As shown in Figure \ref{fig_soc_ext:descriptives_matches_time_attendance}, the number of spectators varies substantially in the various divisions of the German football league system. Columns 3-5 shows the effect of games by the league in which the teams play. The largest effect can be found for the highest league, the \textit{Bundesliga}, in which one game leads to an increase in the assault rate by on average 3.530 assaults per million population. The impact of first division matches is significantly larger than the effect of matches in the other two leagues.
Summing up, there is evidence that the prominence of games is a crucial determinant of the effect size of football matches on the assault rate, with high-rivalry matches and top league games leading to more assaults.













% Robustness
%--------------------------------------------------------------------
\subsection{Robustness Tests}
I perform several sensitivity and placebo tests to assess the robustness of the findings. The results of these checks are reported in Table \ref{tab_soc_ext:reg_fe_assrate_robustness}. Overall, the sensitivity tests demonstrate that the main results are robust to alternative specifications and estimations, indicating that football games do indeed lead to more assaults.\newline

% table Robustness: ECONOMETRICS & other VIOLENCE
\begin{table}[H] \centering 
	\begin{threeparttable} \centering \caption{Robustness tests: Impact on \textbf{Assault rate}}\label{tab_soc_ext:reg_fe_assrate_robustness}
		{\def\sym#1{\ifmmode^{#1}\else\(^{#1}\)\fi} 
			\begin{tabular}{l*{4}{c}}
				\toprule 
				&\multicolumn{1}{c}{(1)}&\multicolumn{1}{c}{(2)}&\multicolumn{1}{c}{(3)}&\multicolumn{1}{c}{(4)}\\
				& coefficient & \clb{c}{standard\\error} & \clb{c}{Effect\\ size [\%]} & $N$ \\
				\midrule
				\textbf{Baseline}                           &2.677\sym{***} &(0.284)    &21.50  & 88,028 \\
\\
\textbf{Econometric specification}\\
\hspace{10pt} Drop delayed games           &2.699\sym{***} &(0.288)    &21.67  & 87,928 \\
\hspace{10pt} No population weights        &5.342\sym{***} &(0.566)    &42.90  & 88,028 \\
\hspace{10pt} Poisson model$^1$			   &1.483\sym{***} &(0.238)    &28.55  & 87,475 \\
\\
\textbf{Other forms of violence}\\
\hspace{10pt} Broadly defined assaults     &5.417\sym{***} &(0.529)    &27.76  & 88,028 \\
\hspace{10pt} Threats                      &0.216\sym{***} &(0.079)    & 5.44  & 88,028 \\
\hspace{10pt} Resistance to enforcement    &0.775\sym{***} &(0.103)    &45.50  & 88,028 \\

 
				\bottomrule 
		\end{tabular}}
		\begin{tablenotes} 
			\item \scriptsize \emph{Notes:} The specifications use daily data (excluding June) spanning the time window 2011-2015 for regions that host games of a soccer team from the first two leagues of the German football league system. The outcome variable is defined as the number of assaults per million population. Except where otherwise noted, the specifications use population-weights. Days are defined to run from 6:00 AM until 5:59 AM the following day to accommodate the fact that offenses committed in the early hours have their origin in the preceding day. The effect size corresponds to the percent change of the assault rate due to a soccer game in relation to the mean when no game takes place. All specifications use region and date fe, their interactions, weather controls, and holiday FE. Two-way clustered standard errors at region-year and year-month level are reported in parentheses. \newline Significance levels: * p < 0.10, ** p < 0.05, *** p < 0.01.\newline \hspace*{15pt} $^1$: number of assaults is dependent variable.
		\end{tablenotes} 
	\end{threeparttable} 
\end{table}


% Alternative econometrics: sample, weights, estimation
\textbf{Alternative Econometric Specifications.---}First, I show that the results are not sensitive to alterations in the \textit{sample}. Adjustments to the sample may be necessary as some of the games are played on different days than originally planned. Deviation from the original match schedule may pose a risk to the allocation of games that is plausibly random. For this reason, I exclude the set of rescheduled games from the analysis. The results are almost identical to the baseline results.\footnote{Of the 4,461 games in the sample, 2.24 percent are rescheduled. The vast majority of rescheduled matches (95 percent) take place in league three.} Second, I show the effects of omitting population \textit{weights} from the regressions. The effect of football matches without considering population weights is considerably larger: a home game leads to an increase in the average assault rate by 5.342 assaults per million population. The higher coefficient compared to the baseline specification indicates that the effects of rural areas must be larger, as they become more relevant when population weights are omitted. Third, I show that my results are robust to a different \textit{estimation} procedure and an alternative measure of the \textit{dependent variable}. In the following, I use the raw number of assaults as the outcome variable. Given the discrete nature of the dependent variable and the fact that there are many cells with zero reports, I employ a Poisson model. The corresponding regression specification reads as follows:
\begin{align}
E\left[ \text{Assaults}_{rdmy}\right.&|\left.\text{Gameday}_{rdmy},\vartheta_r,\text{date}_{dmy}, \mathbf{X}_{rdmy} \right] \nonumber \\ &= exp\left( \beta\ (\text{Gameday}_{rdmy}) + \vartheta_r + \text{date}_{dmy} + \lambda\mathbf{X}_{rdmy} \right). 
\label{eq_soc_ext:poisson_model}
\end{align}
Due to the nonlinearity of the model, the coefficient in Table \ref{tab_soc_ext:reg_fe_assrate_robustness} shows average marginal effects of a home game on the number of physical assaults.\footnote{In nonlinear models, coefficients cannot be interpreted as marginal effects. The partial effect for a Poisson model is given by $\frac{\partial E\left[ \text{Assaults}| \mathbf{X}\right]}{\partial x_j} = \beta_j\ exp(\mathbf{x'}\beta)$. In order to present a single response value, I follow \cite{cameron2005microeconometrics} and report the average response: $\frac{1}{N}\sum_i \frac{\partial E\left[ \text{Assaults}_i| \mathbf{X}_i\right]}{\partial x_{ij}} = \hat\beta_j \times \frac{1}{N} \sum_i\ exp(\mathbf{x}_i'\hat\beta)$.} A home game is predicted to lead to an average increase of 1.483 assaults. This represents a 28.55 percent increase in the number of assaults. \newline

% fig: PLACEBO GAMES
\begin{figure}[H]\centering
	\caption{The effect of placebo games}\label{fig_soc_ext:robustness_placebo_games}
	\includegraphics[width=0.9 \linewidth]{regression/robustness_placebo_games.pdf}
	\begin{minipage}{0.95\linewidth}
		\scriptsize{\emph{Notes:} The figure shows the effect of placebo games. Panel A presents the distribution of the coefficients (along with a normal density) after 10,000 iterations. Panel B shows the distribution of the t-statistics and the resulting ranges of significant coefficients, with a level of significance $\alpha=0.05$. Panel C shows the fraction of significant estimates across the number of iterations.}
	\end{minipage}
\end{figure}
% Other forms of violence 
\textbf{Other Forms of Violence.---}Next, I investigate the robustness of the findings when considering at other forms of violence. First, a broader definition of assaults is considered. In addition to offenses coded as `simple willful bodily harm', I further include negligent, dangerous, and grievous bodily harm as well as brawls. Appendix Table \ref{tab_soc_ext:offense_keys_code} illustrates which offenses are included in the expanded definition of assaults and how the penal codes of the German Criminal Code (\textit{StGB}) are translated into the offense keys of the Police Crime Statistics. The effect size of 5.417 assaults per million population is significantly larger than the baseline coefficient. This indicates that football games also lead to an increase in other forms of physical violence. Second, I investigate the effect of football games on threats, which can be regarded as a precursor of physical violence. The effect size of 0.216 threats per million population is relatively small. The estimate represents a 5.4 percent increase. Third, offenses coded as resisting law enforcement officers are examined. I find that football games lead to on average 0.775 offenses per million population. Overall, the results of these robustness checks indicate that other forms of violence are also affected by professional football games. This suggests that there is no substitution of different types of crime, but that additional violent offenses are committed.\newline



% Placebo games
\textbf{Placebo Games.---}I estimate the impact of \textit{placebo games} on the assault rate to test whether the previous results are only due to chance.\footnote{Unfortunately, there are no offenses that can function as placebo outcomes. This is because most of the offenses covered in the PCS are potentially affected by football games.} The actual matches take place on about five percent of the days in the sample. To estimate the effect of placebo games, I drop the affected days with the actual matches and randomly assign dummy indicators with the same frequency of the real matches. Subsequently, I estimate the model as shown in equation \ref{eq_soc_ext:model}. This procedure is carried out 10,000 times and the results are shown in Figure \ref{fig_soc_ext:robustness_placebo_games}. Panel A displays the distribution of the coefficients. As expected, the coefficients are centered around zero. Panel B illustrates the distribution of the t-statistics. The red area below the kernel density indicates significant estimates for a significance level of $\alpha=0.05$. Panel C shows that with 10,000 iterations, 6.57 percent of the estimates are significantly different from zero. At a significance level of $\alpha=0.01$, there are 3.06 percent significant estimates. The low levels of significant coefficients confirm that the previous results are not due to chance. 













%--------------------------------------------------------------------
% CONCLUSION
%-------------------------------------------------------------------
\bigskip
\section{Discussion and Conclusion}\label{sec_soc_ext:conclusion}

% summary
In this paper, I analyze the impact of professional football games on violent behavior. To estimate the causal effects of football games on physical assaults, I use a generalized difference-in-differences approach that exploits variation in the timing of matches. I compare regional assault rates on days with and without matches, conditional on the day of the week, month, and year while additionally accounting for potential confounding variation coming from weather and holidays. I match web-scraped information on 4,461 football matches with data on local assault rates, weather, holidays, and population figures to construct a panel at the municipality-day level for the period 2011-2015. I find that a home game increases the average assault rate by 21.5 percent. Male victimization rates drive the results and are particularly high for the 18-29 age group. Besides, the the effects are larger for victims with no prior relationship to the suspect and for completed offenses. There are no offsetting reductions in assault rates on days adjacent to game days or in nearby regions. In examining potential channels, I find no evidence that emotional cues are responsible for the increase in assaults on game days. Although there is no evidence supporting the frustration-aggression hypothesis, I find large effects for prominent games and teams. In fact, this set of results suggests that spectators mimic player behavior, select into specific matches, or that the mere agglomeration of fans leads to an increase in the assault rates on game days. 




% policy perspective
I find the external effects of football games on violent behavior to be large in magnitude and economically relevant. For instance, professional football games explain 17.7 percent of all assault reports in the regions in which professional football clubs are located. Back-of-the-envelope calculations indicate that football games in the top three leagues of the German league system precipitate an additional 18,770 assaults in the 2014/15 season,\footnote{These calculations are based on an estimated reduction in assaults per million population per day, 335 days of the football season, and a population of 20,93 million in the affected regions. The number of prevented assaults for the 2014/15 season is: $2.677 \times 335 \times \frac{20,930,000}{1,000,000} = 18,770.$} which translate into annual social cost of 95 million euros.\footnote{To calculate the annual social cost, I use an estimated cost of 5,067 euros (in 2020 prices) for one assault \citep{glaubitz2016kostet}.} The \cite{coase1960problem} theorem states that market failures resulting from externalities can be solved without public intervention (taxpayers do not have to pay for police operations) under the following conditions: clearly defined property rights, complete information, and low/no transaction costs. The question is whether the negative externalities would be internalized in absence of state action. To this end, the football clubs must (i) have an incentive to prevent violence at the events and (ii) be able to implement the incentive effectively \citep{daumann2012}. As for the former, the clubs have a credible interest in maintaining a peaceful atmosphere. Otherwise fans, sponsors, and media might choose to avoid the matches. Moreover, clubs can either employ private security companies or reimburse the costs of police operations to effectively prevent violence in and around the stadiums. Therefore, it should be in the clubs' own interest to provide funding for security. Critics might argue that this procedure is incompatible with the constitution, as public safety is a sovereign right and obligation. However, from an economic perspective, it is helpful to distinguish between stadium grounds and public space when discussing the reimbursement of costs for police operations \citep{mause2020}. On stadium grounds, a football match can be considered a private good.\footnote{The attendance of a football match can be regarded as a private good as both conditions of rivalry (each seat can only be sold once) and excludability (the host can deny some people access) are satisfied.} Accordingly, consumers and producers should pay to ensure security. In contrast, socialization of police costs is justifiable in the public domain as all members of the society benefit equally. This regulation, the assumption of police costs on the club premises, is consistent with the statutes in France, Switzerland, and Great Britain. From the point of view of welfare economics, both positive and negative externalities need to be considered and it must be analyzed whether football matches increase social welfare. However, many positive effects of football matches are not externalities in the strict sense as they have already been internalized.\footnote{Many economic agents have to pay a fee to be included in the value chain of professional football, such as television companies that pay fees to be allowed to broadcast the sporting events.} Therefore, an interesting task for future research is to assess all relevant externalities, positive and negative alike, and evaluate whether professional football games generate a positive social net utility. 

