%--------------------------------------------------------------------
%	DOCUMENT CLASS
%--------------------------------------------------------------------
\documentclass[11pt, a4paper]{article} % type of document (paper, presentation, book,...); scrartcl class with sans serif titles, European layout 
\usepackage{fullpage} % leaves less space at margins of page
\usepackage[onehalfspacing]{setspace} % determine line pitch to 1.5

%--------------------------------------------------------------------
%	INPUT
%--------------------------------------------------------------------
\usepackage[T1]{fontenc} 	% Use 8-bit encoding that has 256 glyphs
\usepackage[utf8]{inputenc} % Required for including letters with accents, Umlaute,...
\usepackage{float} 			% better control over placement of tables and figures in the text
\usepackage{graphicx} 		% input of graphics
\usepackage{xcolor} 		% advanced color package
\usepackage{url, hyperref} 	% include (clickable) URLs
\usepackage{pdfpages}		% insert pages of external pdf documents
\setlength{\parskip}{0em}	% vertical spacing for paragraphs
\setlength{\parindent}{0em}	% horizonzal spacing for paragraphs
\usepackage{tikz}
\usepackage{tikzscale}		% helps to adjust tikz pictures to textwidth/linewidth
\usetikzlibrary{decorations.pathreplacing}
\usetikzlibrary{patterns}
\usetikzlibrary{arrows}
\usepackage{eurosym}		% Eurosymbol

% Have sections in TOC, but not in text
\usepackage{xparse}% for easier management of optional arguments
\ExplSyntaxOn
\NewDocumentCommand{\TODO}{msom}
{
	\IfBooleanF{#1}% do nothing if it's starred
	{
		\cs_if_eq:NNT #1 \chapter { \cleardoublepage\mbox{} }
		\refstepcounter{\cs_to_str:N #1}
		\IfNoValueTF{#3}
		{
			\addcontentsline{toc}{\cs_to_str:N #1}{\protect\numberline{\use:c{the\cs_to_str:N #1}}#4}
		}
		{
			\addcontentsline{toc}{\cs_to_str:N #1}{\protect\numberline{\use:c{the\cs_to_str:N #1}}#3}
		}
	}
	\cs_if_eq:NNF #1 \chapter { \mbox{} }% allow page breaks after sections
}
\ExplSyntaxOff

%--------------------------------------------------------------------
%	TABLES, FIGURES, LISTS
%--------------------------------------------------------------------
\usepackage{booktabs} 		% better tables
\usepackage{longtable}		% tables that may be continued on the next page
\usepackage{threeparttable} % add notes below tables
\renewcommand\TPTrlap{}		% add margins on the side of the notes
	\renewcommand\TPTnoteSettings{%
	\setlength\leftmargin{5 pt}%
	\setlength\rightmargin{5 pt}%
}
\usepackage[
center, format=plain,
font=normalsize,
nooneline,
labelfont={bf}
]{caption} 				% change format of captions of tables and graphs 
%USED IN MPHIL: \usepackage[labelfont=bf,labelsep = period, singlelinecheck=off,justification=raggedright]{caption}, other specifications which are nice: labelformat = parens -> number in paranthesis 


%\usepackage{threeparttablex} % for "ThreePartTable" environment, helps to combine threepart and longtable

% Allow line breaks with \\ in column headings of tables
\newcommand{\clb}[3][c]{%
	\begin{tabular}[#1]{@{}#2@{}}#3\end{tabular}}

% allow line breaks with \\ in row titles
\usepackage{multirow}

\newcommand{\rlb}[3][c]{%
\multirow{2}{*}{\begin{tabular}[#1]{@{}#2@{}}#3\end{tabular}}}% optional argument: b = bottom or t= top alignment


\usepackage[singlelinecheck=on]{subcaption}%both together help to have subfigures
\usepackage{wrapfig}				% wrap text around figure


\usepackage{rotating}				% rotating figures & tables
\usepackage{enumerate}				% change appearance of the enumerator
\usepackage{paralist, enumitem}		% better enumerations
\setlist{noitemsep}					% no additional vertical spacing for enurations
%--------------------------------------------------------------------
%	MATH
%--------------------------------------------------------------------
\usepackage{amsmath,amssymb,amsfonts} % more math symbols and commands
\let\vec\mathbf				 % make vector bold, with no arrow and not in italic

%--------------------------------------------------------------------
%	LANGUAGE SPECIFICS
%--------------------------------------------------------------------
\usepackage[american]{babel} % man­ages cul­tur­ally-de­ter­mined ty­po­graph­i­cal (and other) rules, and hy­phen­ation pat­terns
\usepackage{csquotes} % language specific quotations

%--------------------------------------------------------------------
%	BIBLIOGRAPHY & CITATIONS
%--------------------------------------------------------------------
\usepackage{csquotes} % language specific quotations
\usepackage{etex}		% some more Tex functionality
\usepackage[nottoc]{tocbibind} %add bibliography to TOC
\usepackage[authoryear, round, comma]{natbib} %biblatex

%--------------------------------------------------------------------
%	PATHS
%--------------------------------------------------------------------
\makeatletter
\def\input@path{{../../analysis/output/tables/}}	%PATH TO TABLES
%or: \def\input@path{{/path/to/folder/}{/path/to/other/folder/}}
\makeatother
\graphicspath{{../../analysis/output/graphs/}}		% PATH TO GRAPHS

%--------------------------------------------------------------------
%	LAYOUT
%--------------------------------------------------------------------
\usepackage[left=3cm,right=3cm,top=2cm,bottom=3cm]{geometry}
\usepackage{pdflscape} % lscape.sty Produce landscape pages in a (mainly) portrait document.

\definecolor{darkblue}{rgb}{0.0,0.0,0.6}
\newcommand\natalia[1]{\textcolor{orange}{#1}}

% CAPTIAL LETTERS FOR SECTION CAPTIONS
%\usepackage{sectsty}
%\sectionfont{\normalfont\scshape\centering\textbf}
%\renewcommand{\thesection}{\Roman{section}.}
%\renewcommand{\thesubsection}{\Alph{subsection}.}%\thesection\Alph{subsection}.
%\subsectionfont{\itshape}
%\subsubsectionfont{\scshape}
%\newcommand\relphantom[1]{\mathrel{\phantom{#1}}}
%\setlength\topmargin{0.1in} \setlength\headheight{0.1in}
%\setlength\headsep{0in} \setlength\textheight{9.2in}
%\setlength\textwidth{6.3in} \setlength\oddsidemargin{0.1in}
%\setlength\evensidemargin{0.1in}

\hypersetup{
  colorlinks  = true,
  citecolor   = darkblue,
 	linkcolor   = darkblue,
  urlcolor    = darkblue 
} % macht die URLS blau   
     
\usepackage{lettrine}	% First letter capitalized

% have date in month year format (i.e. omit the day in dates)
\usepackage{datetime}
\newdateformat{monthyeardate}{%
  \monthname[\THEMONTH], \THEYEAR}
%--------------------------------------------------------------------
%	AUTHOR & TITLE
%--------------------------------------------------------------------
\title{Crime externalities from soccer games in Germany\footnote{I am very grateful to Helmut Rainer for valuable comments and discussions. The author gratefully acknowledges financial support from the Leibniz association. Dominik Ammon provided excellent research assistance. All errors and omissions are my own.
}}
\author{
	Marc Fabel 
		\thanks{Munich Graduate School of Economics (MGSE) and ifo Institute for Economic Research, ifo Center for Labor and Demographic Economics (email: \href{mailto:fabel@ifo.de}{fabel@ifo.de}).
		}
}

\date{\monthyeardate\today}








%--------------------------------------------------------------------
%	BEGIN DOCUMENT
%--------------------------------------------------------------------
\begin{document}
\setcounter{page}{0}  
% \tableofcontents
\newpage
\setcounter{page}{1}    
\maketitle

%\textbf{\color{red} Preliminary and incomplete draft\newline Please do not cite or circulate without the author's permission}
%\renewcommand{\abstractname}{\vspace{-\baselineskip}} % GET RID OF ABSTRACT TITLE

%  \begin{abstract}\noindent 
%   \footnotesize{\begin{center}\textbf{Abstract}\end{center} Place abstract here}
%    \end{abstract}

\bigskip
\tableofcontents

\newpage


%--------------------------------------------------------------------
% INTRODUCTION
%--------------------------------------------------------------------
\section{Introduction}\label{sec_soc_ext:introduction}






%--------------------------------------------------------------------
% BACKGROUND
%--------------------------------------------------------------------
\bigskip
\section{Background}\label{sec_soc_ext:background}

German football league system, different leagues/divisions
https://www.dfl.de/en/about/history/
Bundesliga is an independent body , sperated from the German Football Association (DFL)

%in league 1 and 2; in each 17 home and away games
%matchplan: grundlage internationale Spiele
%In die Vorbereitung der Spielpläne für die Bundesliga und die 2. Bundesliga fließen Vorgaben von Kommunen, Sicherheitsorganen, der Zentralen Informationsstelle Sporteinsätze (ZIS), von internationalen Fußballverbänden (FIFA/UEFA), von Fans, Clubs und Stadionbetreibern ein. Neben naheliegenden Einschränkungen wie der Tatsache, dass benachbarte Clubs versetzt zu ihren Heimspielen antreten sollten, müssen u. a. auch Feiertage, parallele Großveranstaltungen (zB Oktoberfest) oder Spieltermine internationaler Wettbewerbe (UEFA Champions League & UEFA Europa League) berücksichtigt werden.
%https://www.bundesliga.com/de/bundesliga/news/so-entsteht-der-spielplan-bundesliga-2-bundesliga-dfl-computer-software.jsp

%--------------------------------------------------------------------
% IDENTIFICATION
%--------------------------------------------------------------------
\newpage
\section{Empirical strategy}\label{sec_soc_ext:empirical_strategy}




\begin{align}
	Crime_{rcdwmy} = \alpha + \beta Gameday_{rcdwmy} + \underbrace{\gamma_d + \delta_w +  \eta_m + \theta_y}_{date_{dwmy}} + \vartheta_r + X_{rdwmy} \lambda + \varepsilon_{rcdwmy}
	\label{eq_soc_ext:model}
\end{align}
with Crime is crime rate in region $r$, associated with club $c$, on day-of-the-week $d$, in week-of-year $w$, in month $m$ and year $y$. Number of assaults per 100,000 individuals times 365. $X_{rdwmy}$ contains school and public holidays and weather controls.\newline

\underline{Possible interactions:} 
\begin{itemize}
	\item $(\gamma_d \times \delta_m)$: month-specific DOW fixed effects
	\item $r \times time$: e.g. $r \times m$, $r \times w$, $r \times y \times m$,  $r \times y \times w$
\end{itemize}

\underline{Standard errors:}
\begin{itemize}
	\item clustered at club level $c$ (allow for correlation over time within region and across regions corresponding to the same club)
	\item twoway cluster (i) district-year and (ii) year-month
	\item adjust for serial correlation: 
	\begin{itemize}
		\item BDM (2004) clustered at club level
		\item CGM (2008) wild cluster bootstrap t-procedures
	\end{itemize}
\end{itemize}

\underline{drop observations:}
\begin{itemize}
	\item are there games on all DOW, probably yes with league 3
	\item drop never affected regions, i.e. regions that are never in a certain radius to a stadium where a game is taking place
	\item focus on only season (i.e. exclude June and winter break)
	\item how many games are played on holidays? drop these? 
	\item focus on particular age group and gender of victims
	\item drop when information of crime is not available on the hour
\end{itemize}



\underline{other remarks:}
\begin{itemize}
	\item different definition of a day: 6:00 AM - 5:59 AM the next day, see Lindo paper
	\item lags and leads in Gameday see Lindo paper
	\item estimation sample only until June 2015 see Timo's paper?
	\item Spieltag fixed effects?
\end{itemize}



%--------------------------------------------------------------------
% DATA & VARIABLES
%--------------------------------------------------------------------
\newpage
\section{Data}\label{sec_soc_ext:data} 
% Description data

sample spans the period 2011-2015

\subsection{Soccer data}
self-collected, obtained via web-scraping from kicker.de and transfermarkt.de, geographic encoding of the stadiums plus marking of affected districts
65 stadiums, stadiums in which not at least a full season is played are exlcuded from the analysis.\footnote{This affects the stadiums in Lübeck (1), Reutlingen (12), Elversberg (14), and the Air-Berlin stadium in Dusseldorf (3). The number in parenthesis indicates the number of games that were held in that venue. This excludes 0.6\% of all macthes.}

\subsection{Crime data}
German uniform crime reporting program (Polizeiliche Kriminalstatistik) by the Federal Criminal Police Office (Bundeskriminalamt)

There are roughly 120 types of criminal offenses in use (recorded in 6 digit codes), though the vast majority of the cases concentrate on only a handful codes.\footnote{The top 10 of the most prevalent crime keys comprise more than 90\% of the cases.} Figure \ref{fig_soc_ext:offense_types_distribution_2015} shows the distribution of cases per crime key for the twenty most common offense types. The intentional simple bodily injury is by far the most prevalent offense and it contributes to about 45\% of all cases. 


cut the data: temporally Jan 2011 - June 2015, and spatially: only the affected regions

Focus only on assaults.

assaults, 
variation see figure \ref{fig_soc_ext:assault_time_distribution_2014}
per hour: drop cases (roughly 15 \% where there is no information on the hour), the number of assaults increases over the course of the day and peaks around midnight. In order to assign the cases that occur during the early hours to the day, in which they are rooted in, I define a day as 6:00 AM until 5:59 AM. 
Per DOW: considerably higher values on Friday and Saturday night, i.e. the night from Fri to Sat (DO MOST OF THE GAMES HAPPEN THEN?) and SAT to SUN.
Per Month: we see strong seasonality, highest values in May (due to Labor Day), decreases over summer and picks up again towards the colder months. 
per day roughly around 320 per day, peak on new year's eve, high on holidays (carneval) 
\footnote{Per month and per day is only for the year 2014. Appendix Figure  \ref{fig_soc_ext:assault_year_distribution} shows the distribution per day for different years. Not so much difference over the years.}


other characteristics: 40\% of victims are female, average age just above 32 years, and 40\% haven't had any prior relationship with the suspect






\subsection{Weather data}
Deutscher Wetterdienst
reduce sample of weather monitors to the subset of stations that measure all relevent weather variables at the same time. From this set choose the weather monitor with the closest proximity to the stadium. The associated weather monitor to each stadium can be found in figure \ref{fig_soc_ext:map_gem_stadiums_nearest_monitor}. 




The quality of the match between the weather monitors and the stadiums is high, the average distance between stadium and monitor is 15 km. The distribution of the  distance between stadium and monitor can be found in figure 





\subsection{Regional Database}
GENESIS





%--------------------------------------------------------------------
% RESULTS
%--------------------------------------------------------------------
\newpage
\section{Results}\label{sec_soc_ext:results}

\begin{itemize}
	\item general effect of a game (increase number of FE)
	\item separately by home/away game
	\item break it down to hour windows: before, during, and after the game
	\item spillover/displacement effects (temporal \& spatial), important for validity of the identification scheme
	\item \textbf{heterogeneity of the effects:}
	\begin{itemize}
		\item \underline{victim characteristics:} demographics (age, gender), relationship between offender and victim (known/unknown)
		\item \underline{crime characteristics:} timing (season), type of charge (definition of assault), assault execution
		\item \underline{game and club characteristics:} prominent game or team (derby, rivalry, BL/2L/3L), covariate space that was scraped from kicker and transfermarkt, emotional cues (Card \& Dahl story or referee quality, penalty in last minutes,...)
		\item \underline{region characteristics:} Regionaldatenbank - treatment effect heterogeneity (causal forests) - just for this part or for all heterogeneity aspects?
	\end{itemize}
	\item \textbf{robustness:}
	\begin{itemize}
		\item different specification of what are the affected regions, vary radius around stadium which region is affected: at first use only region (gemeinde/Kreis) in which the stadium is located, then play around with radius.
		\item estimation with Poisson? (many zeros and discrete nature of reports - use number of assaults)
		\item placebo games?
		\item control for mass events (see BHR paper)
		\item drop Nachholspiele
	\end{itemize}
\end{itemize}













%--------------------------------------------------------------------
% CONCLUSION
%-------------------------------------------------------------------
\bigskip
\section{Concluding remarks}\label{sec_soc_ext:conclusion}




%--------------------------------------------------------------------
% BIBLIOGRAPHY
%--------------------------------------------------------------------
\newpage


%\bibliographystyle{ecca_edited}%previous style-chicago
%\bibliography{mlch_bibliography}

%\printbibliography


%--------------------------------------------------------------------
% FIGURES AND TABLES
%--------------------------------------------------------------------
%\newpage
%\section{Figures and tables}
\newpage
\TODO\section{Figures}
\vspace*{\fill}
{\Huge \begin{center}\textbf{FIGURES}\end{center}}
\vspace*{\fill}\clearpage
%--------------------------------------------





%WMWMWMWMWMWMWMWMWMWMWMWMWMWMWMWMWMWMWMWM
% DATA
%WMWMWMWMWMWMWMWMWMWMWMWMWMWMWMWMWMWMWMWM

%--------------------------------------------
% crime - key distribution
\vspace*{\fill}
\begin{figure}[H]\centering
	\caption{The types of criminal offenses}\label{fig_soc_ext:offense_types_distribution_2014}
	\includegraphics[width=0.9 \linewidth]{descriptive/soc_ext_offense_key_distribution_2014}
	\begin{minipage}{0.95\linewidth}
		\scriptsize{\emph{Notes:} The figure depicts the frequency distribution of the most common criminal offenses in the Federal Republic of Germany in 2014. The most common offense type is intentional simple bodily injury (224000), followed by threats (232300), and two forms of dangerous and serious bodily injury (222110 \& 222010). These four offense types together comprise around 75\% of all criminal offenses. \newline \emph{Source:} Own representation with data from the Federal Criminal Police Office.}
	\end{minipage}
\end{figure}
\vspace*{\fill}\clearpage
%--------------------------------------------

% crime - assault distribution over time
\vspace*{\fill}
\begin{figure}[H]\centering
	\caption{Distribution of assaults across time}\label{fig_soc_ext:assault_time_distribution_2014}
	\begin{subfigure}[h]{0.48\linewidth}\centering
		\includegraphics[width=\linewidth]{descriptive/soc_ext_assaults_per_hour}
	\end{subfigure}
	\begin{subfigure}[h]{0.48\linewidth}\centering
		\includegraphics[width=\linewidth]{descriptive/soc_ext_assaults_per_dow}
	\end{subfigure}

	\begin{subfigure}[h]{0.48\linewidth}\centering
		\includegraphics[width=\linewidth]{descriptive/soc_ext_assaults_per_month_2014}
	\end{subfigure}
	\begin{subfigure}[h]{0.48\linewidth}\centering
		\includegraphics[width=\linewidth]{descriptive/soc_ext_assaults_per_day_2014_single}
	\end{subfigure}
	\begin{minipage}{\linewidth}
		\scriptsize{\emph{Notes:} The figure shows the distribution of assaults across the course of a day, days of the week, across months (adjusted for the number of days per month), and across the days of the year in the Federal Republic of Germany.\newline \emph{Source:} Own representation with data from the Federal Criminal Police Office.\newline \hspace{1 em} $^1$ The figures in panel C and D are solely based on the year 2014. Please consult the appendix for figures from the other years.}
	\end{minipage}
\end{figure}
\vspace*{\fill}\clearpage
%--------------------------------------------

% WEATHER MONITORS: histogram of the distance stadium <-> monitor
\vspace*{\fill}
\begin{figure}[H]\centering
	\caption{The distance between stadiums and their closest weather monitor}\label{fig_soc_ext:hist_distance_stadium_monitors}
	\includegraphics[width=0.9 \linewidth]{descriptive/soc_ext_distance_monitors_stadiums}
	\begin{minipage}{0.95\linewidth}
		\scriptsize{\emph{Notes:} The figure shows the distribution of the distance between stadiums and their closest weather monitor. The average distance between the two is 15 km. \newline \emph{Source:} Own representation with data from Germany's National Meterological Service (Deutscher Wetterdienst).}
	\end{minipage}
\end{figure}
\vspace*{\fill}\clearpage
%--------------------------------------------


%WMWMWMWMWMWMWMWMWMWMWMWMWMWMWMWMWMWMWMWM
% MAPS
%WMWMWMWMWMWMWMWMWMWMWMWMWMWMWMWMWMWMWMWM
\vspace*{\fill}
\begin{figure}[H]\centering
	\caption{The stadiums}\label{fig_soc_ext:map_gem_stadiums}
	\includegraphics[width=0.9\linewidth]{maps/soc_ext_2010_2015_Gemeinde_stadiums_75.png}
	\begin{minipage}{0.95\linewidth}
		\scriptsize{\emph{Notes:} This map shows the stadiums used in the analysis over the seasons 2010/11 until 2014/15. The black outlines indicate federal state boundaries.\newline \emph{Source:} Own representation with data from the Federal Institute for Research on Building, Urban Affairs and Spatial Development (BBSR).}
	\end{minipage}
\end{figure}
\vspace*{\fill}\clearpage



\vspace*{\fill}
\begin{figure}[H]\centering
	\caption{The stadiums and their closest weather monitors}\label{fig_soc_ext:map_gem_stadiums_nearest_monitor}
	\includegraphics[width=0.9 \linewidth]{maps/soc_ext_2010_2015_Gemeinde_stadiums_nearest_monitor_75.png}
	\begin{minipage}{0.95\linewidth}
		\scriptsize{\emph{Notes:} This map shows the stadiums used in the analysis over the seasons 2010/11 until 2014/15 (red dots) and the closest weather monitors (blue dots). The orange lines indicate how the weather monitors are assigned to the stadiums. The black outlines indicate federal state boundaries.\newline \emph{Source:} Own representation with data from the Federal Institute for Research on Building, Urban Affairs and Spatial Development (BBSR).}
	\end{minipage}
\end{figure}
\vspace*{\fill}\clearpage



% To be put in: 
%	- histogram: distance weather monitors to stadiums
%	- barplot: number of offences, by key


%--------------------------------------------------------------------
% TABLES
%--------------------------------------------------------------------
\newpage
\TODO\section{Tables}
\vspace*{\fill}
{\Huge \begin{center}\textbf{TABLES}\end{center}}
\vspace*{\fill}\clearpage









%--------------------------------------------------------------------
% APPENDIX
%--------------------------------------------------------------------
\newpage
\TODO\section{Appendix}
\vspace*{\fill}
{\Huge \begin{center}\textbf{APPENDIX}\end{center}}
\vspace*{\fill}\clearpage


\renewcommand\thefigure{A\arabic{figure}}
\setcounter{figure}{0} 
\captionsetup[subfigure]{labelformat=parens}



%--------------------------------------------
% crime - assault distribution over time
\vspace*{\fill}
\begin{figure}[H]\centering
	\caption{Distribution of assaults across days of the year, per year}\label{fig_soc_ext:assault_year_distribution}
	\begin{subfigure}[h]{0.48\linewidth}\centering
		\includegraphics[width=\linewidth]{descriptive/soc_ext_assaults_per_day_2011}
	\end{subfigure}
	\begin{subfigure}[h]{0.48\linewidth}\centering
		\includegraphics[width=\linewidth]{descriptive/soc_ext_assaults_per_day_2012}
	\end{subfigure}\begin{subfigure}[h]{0.48\linewidth}\centering
		\includegraphics[width=\linewidth]{descriptive/soc_ext_assaults_per_day_2013}
	\end{subfigure}
	\begin{subfigure}[h]{0.48\linewidth}\centering
		\includegraphics[width=\linewidth]{descriptive/soc_ext_assaults_per_day_2014}
	\end{subfigure}
	\begin{subfigure}[h]{0.48\linewidth}\centering
		\includegraphics[width=\linewidth]{descriptive/soc_ext_assaults_per_day_2015}
	\end{subfigure}
	\begin{minipage}{\linewidth}
		\scriptsize{\emph{Notes:} The figure shows the relative frequency of assaults per day between 2011 and 2015 in the Federal Republic of Germany.\newline \emph{Source:} Own representation with data from the Federal Criminal Police Office}
	\end{minipage}
\end{figure}
\vspace*{\fill}\clearpage
%--------------------------------------------




\end{document}