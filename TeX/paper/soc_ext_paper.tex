%--------------------------------------------------------------------
%	DOCUMENT CLASS
%--------------------------------------------------------------------
\documentclass[11pt, a4paper, draft]{article} % type of document (paper, presentation, book,...); scrartcl class with sans serif titles, European layout 
\usepackage{fullpage} % leaves less space at margins of page
\usepackage[onehalfspacing]{setspace} % determine line pitch to 1.5

%--------------------------------------------------------------------
%	INPUT
%--------------------------------------------------------------------
\usepackage[T1]{fontenc} 	% Use 8-bit encoding that has 256 glyphs
\usepackage[utf8]{inputenc} % Required for including letters with accents, Umlaute,...
\usepackage{float} 			% better control over placement of tables and figures in the text
\usepackage{graphicx} 		% input of graphics
\usepackage{xcolor} 		% advanced color package
\usepackage{url, hyperref} 	% include (clickable) URLs
\usepackage{pdfpages}		% insert pages of external pdf documents
\setlength{\parskip}{0.75em}	% vertical spacing for paragraphs
\setlength{\parindent}{0em}	% horizonzal spacing for paragraphs
\usepackage{tikz}
\usepackage{tikzscale}		% helps to adjust tikz pictures to textwidth/linewidth
\usetikzlibrary{decorations.pathreplacing}
\usetikzlibrary{patterns}
\usetikzlibrary{arrows}
\usepackage{eurosym}		% Eurosymbol

% Have sections in TOC, but not in text
\usepackage{xparse}% for easier management of optional arguments
\ExplSyntaxOn
\NewDocumentCommand{\TODO}{msom}
{
	\IfBooleanF{#1}% do nothing if it's starred
	{
		\cs_if_eq:NNT #1 \chapter { \cleardoublepage\mbox{} }
		\refstepcounter{\cs_to_str:N #1}
		\IfNoValueTF{#3}
		{
			\addcontentsline{toc}{\cs_to_str:N #1}{\protect\numberline{\use:c{the\cs_to_str:N #1}}#4}
		}
		{
			\addcontentsline{toc}{\cs_to_str:N #1}{\protect\numberline{\use:c{the\cs_to_str:N #1}}#3}
		}
	}
	\cs_if_eq:NNF #1 \chapter { \mbox{} }% allow page breaks after sections
}
\ExplSyntaxOff

%--------------------------------------------------------------------
%	TABLES, FIGURES, LISTS
%--------------------------------------------------------------------
\usepackage{booktabs} 		% better tables
\usepackage{longtable}		% tables that may be continued on the next page
\usepackage{threeparttable} % add notes below tables
\renewcommand\TPTrlap{}		% add margins on the side of the notes
	\renewcommand\TPTnoteSettings{%
	\setlength\leftmargin{5 pt}%
	\setlength\rightmargin{5 pt}%
}
\usepackage[
center, format=plain,
font=normalsize,
nooneline,
labelfont={bf}
]{caption} 				% change format of captions of tables and graphs 
%USED IN MPHIL: \usepackage[labelfont=bf,labelsep = period, singlelinecheck=off,justification=raggedright]{caption}, other specifications which are nice: labelformat = parens -> number in paranthesis 


%\usepackage{threeparttablex} % for "ThreePartTable" environment, helps to combine threepart and longtable

% Allow line breaks with \\ in column headings of tables
\newcommand{\clb}[3][c]{%
	\begin{tabular}[#1]{@{}#2@{}}#3\end{tabular}}

% allow line breaks with \\ in row titles
\usepackage{multirow}

\newcommand{\rlb}[3][c]{%
\multirow{2}{*}{\begin{tabular}[#1]{@{}#2@{}}#3\end{tabular}}}% optional argument: b = bottom or t= top alignment


\usepackage[singlelinecheck=on]{subcaption}%both together help to have subfigures
\usepackage{wrapfig}				% wrap text around figure


\usepackage{rotating}				% rotating figures & tables
\usepackage{enumerate}				% change appearance of the enumerator
\usepackage{paralist, enumitem}		% better enumerations
\setlist{noitemsep}					% no additional vertical spacing for enurations
%--------------------------------------------------------------------
%	MATH
%--------------------------------------------------------------------
\usepackage{amsmath,amssymb,amsfonts} % more math symbols and commands
\let\vec\mathbf				 % make vector bold, with no arrow and not in italic

%--------------------------------------------------------------------
%	LANGUAGE SPECIFICS
%--------------------------------------------------------------------
\usepackage[american]{babel} % man­ages cul­tur­ally-de­ter­mined ty­po­graph­i­cal (and other) rules, and hy­phen­ation pat­terns
\usepackage{csquotes} % language specific quotations

%--------------------------------------------------------------------
%	BIBLIOGRAPHY & CITATIONS
%--------------------------------------------------------------------
\usepackage{csquotes} % language specific quotations
\usepackage{etex}		% some more Tex functionality
\usepackage[nottoc]{tocbibind} %add bibliography to TOC
\usepackage[authoryear, round, comma]{natbib} %biblatex

%--------------------------------------------------------------------
%	PATHS
%--------------------------------------------------------------------
\makeatletter
\def\input@path{{../../analysis/output/tables/}}	%PATH TO TABLES
%or: \def\input@path{{/path/to/folder/}{/path/to/other/folder/}}
\makeatother
\graphicspath{{../../analysis/output/graphs/}}		% PATH TO GRAPHS

%--------------------------------------------------------------------
%	LAYOUT
%--------------------------------------------------------------------
\usepackage[left=3cm,right=3cm,top=2cm,bottom=3cm]{geometry}
\usepackage{pdflscape} % lscape.sty Produce landscape pages in a (mainly) portrait document.

\definecolor{darkblue}{rgb}{0.0,0.0,0.6}
\newcommand\natalia[1]{\textcolor{orange}{#1}}

% CAPTIAL LETTERS FOR SECTION CAPTIONS
%\usepackage{sectsty}
%\sectionfont{\normalfont\scshape\centering\textbf}
%\renewcommand{\thesection}{\Roman{section}.}
%\renewcommand{\thesubsection}{\Alph{subsection}.}%\thesection\Alph{subsection}.
%\subsectionfont{\itshape}
%\subsubsectionfont{\scshape}
%\newcommand\relphantom[1]{\mathrel{\phantom{#1}}}
%\setlength\topmargin{0.1in} \setlength\headheight{0.1in}
%\setlength\headsep{0in} \setlength\textheight{9.2in}
%\setlength\textwidth{6.3in} \setlength\oddsidemargin{0.1in}
%\setlength\evensidemargin{0.1in}

\hypersetup{
  colorlinks  = true,
  citecolor   = darkblue,
 	linkcolor   = darkblue,
  urlcolor    = darkblue 
} % macht die URLS blau   
     
\usepackage{lettrine}	% First letter capitalized

% have date in month year format (i.e. omit the day in dates)
\usepackage{datetime}
\newdateformat{monthyeardate}{%
  \monthname[\THEMONTH], \THEYEAR}
%--------------------------------------------------------------------
%	AUTHOR & TITLE
%--------------------------------------------------------------------
\title{Crime Externalities from Soccer Games in Germany\footnote{I am very grateful to Helmut Rainer for valuable comments and discussions. The author gratefully acknowledges financial support from the Leibniz association. Dominik Ammon provided excellent research assistance. All errors and omissions are my own.
}}
\author{
	Marc Fabel 
		\thanks{Munich Graduate School of Economics (MGSE) and ifo Institute for Economic Research, ifo Center for Labor and Demographic Economics (email: \href{mailto:fabel@ifo.de}{fabel@ifo.de}).
		}
}

\date{\monthyeardate\today}








%--------------------------------------------------------------------
%	BEGIN DOCUMENT
%--------------------------------------------------------------------
\begin{document}
\setcounter{page}{0}  
% \tableofcontents
\newpage
\setcounter{page}{1}    
\maketitle

%\textbf{\color{red} Preliminary and incomplete draft\newline Please do not cite or circulate without the author's permission}
%\renewcommand{\abstractname}{\vspace{-\baselineskip}} % GET RID OF ABSTRACT TITLE

%  \begin{abstract}\noindent 
%   \footnotesize{\begin{center}\textbf{Abstract}\end{center} Place abstract here}
%    \end{abstract}

\bigskip
\tableofcontents

\newpage


%--------------------------------------------------------------------
% INTRODUCTION
%--------------------------------------------------------------------
\section{Introduction}\label{sec_soc_ext:introduction}






%--------------------------------------------------------------------
% BACKGROUND
%--------------------------------------------------------------------
\bigskip
\section{Background}\label{sec_soc_ext:background}

German football league system, different leagues/divisions
https://www.dfl.de/en/about/history/
Bundesliga is an independent body , sperated from the German Football Association (DFL)

%in league 1 and 2; in each 17 home and away games
%matchplan: grundlage internationale Spiele
%In die Vorbereitung der Spielpläne für die Bundesliga und die 2. Bundesliga fließen Vorgaben von Kommunen, Sicherheitsorganen, der Zentralen Informationsstelle Sporteinsätze (ZIS), von internationalen Fußballverbänden (FIFA/UEFA), von Fans, Clubs und Stadionbetreibern ein. Neben naheliegenden Einschränkungen wie der Tatsache, dass benachbarte Clubs versetzt zu ihren Heimspielen antreten sollten, müssen u. a. auch Feiertage, parallele Großveranstaltungen (zB Oktoberfest) oder Spieltermine internationaler Wettbewerbe (UEFA Champions League & UEFA Europa League) berücksichtigt werden.
%https://www.bundesliga.com/de/bundesliga/news/so-entsteht-der-spielplan-bundesliga-2-bundesliga-dfl-computer-software.jsp

%--------------------------------------------------------------------
% IDENTIFICATION
%--------------------------------------------------------------------
\newpage
\section{Empirical strategy}\label{sec_soc_ext:empirical_strategy}

In order to identify the causal effect of soccer matches on criminal behavior, I exploit time series and cross-sectional variation. To be precise, I compare the regional assault rate on the game day to the expected assault level conditional on the day of the week, month, and year while additionally accounting for other possible confounding variation stemming from weather and holidays. In other words, the counterfactual assault rate (what would be expected on a game day in absence of the game), e.g. a Saturday in April 2012, is obtained by using the assault rate on other Saturdays in April 2012, where no games take place. %In other words, the regional assault rate on a given date without a game (say a Saturday in April 2012) serves as counterfactual for what would be expected on a game day (on a Saturday in April 2012) in absence of the game. 


The identification strategy is based on a generalized difference-in-differences approach to study the impact of a soccer match on violent behavior. Let Assaults$_{rdmy}$ denote the assault rate in region $r$, on day-of-the-week $d$, in month $m$ and year $y$, which in the simple baseline model is given by:\footnote{For ease of comparison, the assault rate is defined as the number of assaults per 100,000 individuals multiplied with the number of days per year. For details, refer to \cite{hener2019noise}.}

\begin{align}
	\text{Assaults}_{rdmy} = \alpha + \beta\ (\text{Gameday}_{rdmy}) + \vartheta_r + \underbrace{\gamma_d +  \eta_m + \theta_y}_{\text{date}_{dmy}} + \lambda \mathbf{X}_{rdmy} + \varepsilon_{rdmy}
	\label{eq_soc_ext:model}.
\end{align}

$\text{Gameday}_{rdmy}$ is a binary variable equal to unity when a home games takes place and zero otherwise. Region fixed effects $\vartheta_r$ capture time-invariant differences between regions and ensure that the identification is driven by within instead of between region variation over time. The vector $\text{date}_{dmy}$ contains fixed effects for day-of-the-week ($\gamma_d$), month ($\eta_m$), and year ($\theta_y$). This way, the model flexibly controls for day-of-week specific heterogeneity, seasonal effects and long-run time trends. I expand on the baseline model by adding interactions of region fixed effects with all elements of $\text{date}_{dmy}$, i.e. region-by-day-of-week fixed effects, region-by-month fixed effects, and region-by-year fixed effects. The interactions account for systemic changes in the degree of violent behavior over the year for each region. The vector $\mathbf{X}_{rdmy}$ includes school and public holidays and weather controls.\footnote{Holiday controls include binary variables (on the level of the Federal States) for All Saints, Ascension Day, Assumption Day, Christmas, Corpus Christi, Epiphany, Easter, German Unity Day, Good Friday, Labor Day, New Year, Penance Day, Pentecost, and Reformation Day. Moreover, it contains dummy variables for carnival season and New Year's Eve.\newline Weather controls (on the regional level) include average air temperature, maximum air temperature, minimum air temperature, minimum ground temperature, steam pressure, cloud cover, air pressure, humidity, average precipitation, sunshine duration, snow depth, and wind speed-.} I compute two-way cluster-robust standard errors to capture arbitrary correlation at the region-year and year-month levels. Observations are weighted using population figures from the Federal Statistical Office.


The implicit assumption to interpret the parameter of interest $\beta$ as the causal effect of a home game on violent behavior, is that the location and the time of a soccer match is orthogonal to the number of assaults, conditional on the covariates. However, displacement effects may pose a threat to identification. On the one hand, this refers to spatial displacement effects, which may occur when (violence-prone) populations come from distant regions to the game. On the other hand, this includes temporal displacement effects, which happen when assaults from neighbouring days are shifted to the game day. In both cases, the parameter would overestimate the impact of a soccer match on violent behavior as the offense would would have been committed anyway, but just another time or at another place.


Adress the challenges by X and X. 
I find bla


XXX only specification of home game: when there is an away game this could mean particularly aggresive fans are out of town, which might decrease the assault rate. check for that (see Helmut's comments on that)
I find bla






\newpage
\subsection{Idea collection}
\underline{Further Possible interactions:} 
\begin{itemize}
	\item $(\gamma_d \times \delta_m)$: month-specific DOW fixed effects
	\item $r \times time$: e.g. $r \times w$, $r \times y \times m$,  $r \times y \times w$
\end{itemize}

\underline{Other approaches to compute standard errors:}
\begin{itemize}
	\item clustered at club level $c$ (allow for correlation over time within region and across regions corresponding to the same club)

	\item adjust for serial correlation: 
	\begin{itemize}
		\item BDM (2004) clustered at club level
		\item CGM (2008) wild cluster bootstrap t-procedures
	\end{itemize}
\end{itemize}

\underline{drop observations:}
\begin{itemize}
	\item focus on only season (i.e. exclude winter break)
	\item how many games are played on holidays? drop these? 
	\item drop when information of crime is not available on the hour
\end{itemize}



%--------------------------------------------------------------------
% DATA & VARIABLES
%--------------------------------------------------------------------
\newpage
\section{Data}\label{sec_soc_ext:data} 
% Description data
The data set used for the analysis covers the time window 2011-2015 and contains the regions in which soccer games of the first three leagues of the German football league system take place. I combine five data sources in order to examine the impact of public mass events on criminal behavior.



\subsection{Crime data}
The crime data is coming from the German Police Crime Statistics (Polizeiliche Kriminalstatistik), which is provided by the Federal Criminal Police Office (Bundeskriminalamt).\footnote{Many aspects of the data preparation is inspired from \cite{hener2019noise} who is using the same data in order to examine the causal effect of noise pollution on criminal behavior.} It covers the universe of individuals that fall victim to a crime against their legally protected personal rights between 2011 and 2015. Yet, as the data is only reported after the end of police procedures, the data is truncated to the period from January 2011 to May 2015 to avoid any problems with lags between the occurrence of the crime and when it is reported. The data includes next to the time (up to the hour) and place (municipality level) of the crime, the crime type code, the victim's age and gender, characteristics of the assault execution (attempt/successful, usage of firearm, lone operator/crime was committed by a group) and information on the relationship between victim and suspect.\footnote{The relationship between victim and suspect is retrieved in two ways. On the one hand, this covers formal relationships (such as types of kin or acquaintances). On the other hand, this examines relationships in a spatial-social manner (for instance living in the same household, or being in an education or care context).} Roughly 40\% of the victims are female, their average age is 32 years, and 40\% of the victims haven't had any prior relationship with the suspect.

The micro-data is aggregated to the municipality-day level and the main outcome, the assault rate, is defined as the number of assaults per 100,000 inhabitants, and multiplied by 365 in order to mirror annual crime rates. Assaults are defined as actions involving physical violence. For that purpose, I use the crime type code of 'intentional simple bodily injury' (Vorsätzliche einfache Körperverletzung § 223 StGB). There are roughly 120 types of criminal offenses in use (recorded in 6 digit codes), though the vast majority of the cases concentrate on only a handful of codes.\footnote{The top 10 of the most prevalent crime keys comprise more than 90\% of the cases.} Figure \ref{fig_soc_ext:offense_types_distribution_2014} shows the distribution of cases per crime key for the twenty most common offense types in 2014. The intentional simple bodily injury is by far the most prevalent offense and it contributes to about 45\% of all cases. The prevalence of the offense type and its association with aggression motivates the decision to have assaults in this definition as the main outcome variable.

Figure \ref{fig_soc_ext:assault_time_distribution_2014} shows the distribution of assaults over time. Panel A displays the variation of assaults per hour of the day.\footnote{Roughly 15\% of the observations do not contain information on the hour. This does not entail any consequences for the main analysis, as I examine daily variation in the assault rate.} The number of assaults is increasing over the course of a day and peaks around midnight. In order to assign the cases that occur during the early hours to the day, in which they are rooted in, I define a day as running from 6:00 AM until 5:59 AM. Panel B shows the distribution of assaults across days of the week. You can see higher values on Friday and Saturday, whereas the other days have assault rates belonging to the same ballpark. Panel C shows that the number of assaults have a strong seasonal pattern with the highest value recorded in May, and the smallest figure in August.\footnote{Panel C shows the number of assaults across months of the year while adjusting for the number of days per month.} Panel D confirms this impression by plotting the daily number of assaults. New year's eve is a particular impressive outlier in this context. \footnote{Panels C and D are showing data for the year 214 only. Appendix Figure  \ref{fig_soc_ext:assault_year_distribution} shows the distribution per day for different years.}

% XXX per day roughly around 320 per day, peak on new year's eve, high on holidays (carneval) 
\textbf{XXX delete June, as there are no games played in that month}



\subsection{Soccer data}
The data on soccer matches is self-collected and is obtained via web-scraping from kicker.de and transfermarkt.de. It captures all matches that are carried out in the context of the German football league system over the period January 2011 until May 2015. The data contains detailed match and table standing parameters, e.g. time and place of the match, attendance, pregame point difference, goals, penalties, cards, referee characteristics, among others. Furthermore, there is comprehensive data on the single teams; such as size of the team, average age, market value, and number of foreign players. The stadiums where the matches take place are geographically encoded and will "activate" the treatment in that region in which the stadium is located and a home game is played. Figure \ref{fig_soc_ext:map_gem_stadiums} depicts a map with all 69 stadiums in the data set. 

% XXX  stadiums in which not at least a full season is played are excluded from the analysis.\footnote{This affects the stadiums in Lübeck (1), Reutlingen (12), Elversberg (14), and the Air-Berlin stadium in Dusseldorf (3). The number in parenthesis indicates the number of games that were held in that venue. This excludes 0.6\% of all matches.}

Figure \ref{fig_soc_ext:descriptives_matches_time_attendance} illustrates some insights into key variables. Panel A shows the number of matches per day of the week and league. The vast majority of the matches is taking place between Friday and Sunday.\footnote{The inclusion of day-of-week fixed effects in the baseline specification help to account for the higher share of games played on weekends, which are associated with higher levels of criminal behavior.} This is particularly true for the highest league, whereas lower leagues play occasionally on weekdays, too. Matches happening on weekdays are exclusively held in the evenings. In contrast, matches on weekends occur usually in the afternoon. The three different divisions can attract spectators differently, as depicted in Panel C. The Bundesliga as the highest tier can intrigue the most people to show up with an average audience of 44,000 people per game, followed by the second league with on average 17,000 fans per match, and the lowest league can attract on average a bit less than 6,000 fans per game to show up in the stadium.



\subsection{Weather data}
The weather data comes from Germany's National Meteorological Service (Deutscher Wetterdienst). In order to construct the weather control variables, I use the sample of weather monitors, which measure relevant weather variables all at the same time.\footnote{I use the following weather variables as daily averages: the daily average, minimum and maximum air temperature,  minimum ground temperature, vapor pressure, air pressure, cloud cover, air humidity, precipitation, sun shine duration, snow depth, and wind velocity.} From this set of monitors I choose the weather monitor with the closest proximity to a stadium. The assigned monitor-stadium pairs can be found in Figure \ref{fig_soc_ext:map_gem_stadiums_nearest_monitor}. The quality of the matches between weather monitors and stadiums is high, as the average distance between stadiums and monitors is 15 kilometers. The distribution of the distance of monitor-stadium pairs can be found in Figure \ref{fig_soc_ext:hist_distance_stadium_monitors}. Few of the  weather variables have missing data, which are filled in the following way: propagate the last valid observation forward to the next valid observation (i.e. forward fill).\footnote{The weather variables with missing data are (with the share of missing data in parenthesis): Cloud cover (<1.2\%), snow depth (<0.6\%).}



\subsection{Holidays}
In order to capture any variation of the crime rate between ordinary and special days, I add controls for public and school holidays, which may differ on the state level.\footnote{The data for the school holiday comes from The Standing Conference of the Ministers of Education and Cultural Affairs of the Länder in the Federal Republic of Germany (Kultusminister Konferenz).} Furthermore, I also add a dummy variable for peculiar days (New Year's Eve and the days surrounding Carnival), which are not holidays, but certainly shift the crime rate anyways.



\subsection{Regional Database}
The Federal Statistical Office and the statistical offices of the Länder provide a data base of detailed statistics by various subject areas on a very granular spatial level. Thus, I am able to build a panel on the municipality-year level containing comprehensive information on topics such as territory, population, labor market, election results, housing, economic sectors, and public budgets. The information is used to construct weights for the regression analysis or to build assault rates. Furthermore, the plan for the future is to leverage the high dimensionality of the data in order to investigate treatment effect heterogeneity with the help of machine learning algorithms (e.g. by making use of causal forests). 














%--------------------------------------------------------------------
% RESULTS
%--------------------------------------------------------------------
\newpage
\section{Results}\label{sec_soc_ext:results}

\begin{itemize}
	\item general effect of a game (increase number of FE)
	\item separately by home/away game
	\item break it down to hour windows: before, during, and after the game
	\item spillover/displacement effects (temporal \& spatial), important for validity of the identification scheme
	\item \textbf{heterogeneity of the effects:}
	\begin{itemize}
		\item \underline{victim characteristics:} demographics (age, gender), relationship between offender and victim (known/unknown)
		\item \underline{crime characteristics:} timing (season), type of charge (definition of assault), assault execution
		\item \underline{game and club characteristics:} prominent game or team (derby, rivalry, BL/2L/3L), covariate space that was scraped from kicker and transfermarkt, emotional cues (Card \& Dahl story or referee quality, penalty in last minutes,...)
		\item \underline{region characteristics:} Regionaldatenbank - treatment effect heterogeneity (causal forests) - just for this part or for all heterogeneity aspects?
	\end{itemize}
	\item \textbf{robustness:}
	\begin{itemize}
		\item different specification of what are the affected regions, vary radius around stadium which region is affected: at first use only region (gemeinde/Kreis) in which the stadium is located, then play around with radius.

		\item placebo games and placebo outcomes (any type that is not affected)?
		\item control for mass events (see BHR paper)
		\item drop Nachholspiele
	\end{itemize}
\end{itemize}





\newpage
\subsection{Home and away games}
so far only analysis of home games, is that fair? 
away games completetly different: the most devoted fans travel out of the municipality - could this affect the assault rate as well? Is it eventually lower when there is an away game?

In order to answer that I build up my data a bit differently. Whereas in the normal version, the soccer data was merged to the other data on the match level (with the home team's region ID as identifier), in this case I use the soccer data on the table standings level (so both I match both, the home and away team to a region). One consequence of this approach is that it makes the treatment status of single regions less clear. What do you do with regions (that have more than one team) when there is a home and an away game at the same day. In order to alleviate this concern, I drop the third league from the sample and focus only on the first two leagues.\footnote{There remain still a few ambiguities but they are solved the following: in 12\% of the matches there is still a doubling of two teams of one region play on the same day, either home and away game, or two away games. The status of the region on that day in the latter case is set to away. In the former case, I put it to home game.} This approach helps to clarify the treatment status of a region considerably.

Table \ref{tab_soc_ext:reg_fe_assrate_home_away}


to make it fair and square column 1 shows the baseline model (considers only home games) when dropping league three from the original sample and adjust the games as it is done in the distinction of home/away. 

Column 2 presents the estimates when allowing home and game effects to have an impact on the assault rate. The effect of 


% \clb{c}{prior\\relationship} 
% table home & away games
%--------------------------------------------
\vspace*{\fill}
\begin{table}[H] \centering 
	\begin{threeparttable} \centering \caption{Effect of soccer games on the assault rate, distinction of home and away game}\label{tab_soc_ext:reg_fe_assrate_home_away}
		{\def\sym#1{\ifmmode^{#1}\else\(^{#1}\)\fi} 
			\begin{tabular}{l*{2}{c}}
				\toprule 
				&\multicolumn{1}{c}{(1)}&\multicolumn{1}{c}{(2)}\\
				&\clb{c}{baseline\\w/o L3} & \clb{c}{distinction\\home/away$^1$}\\
				\midrule
				Game day            &     104.382\sym{***}&                     \\
                    &    (11.834)         &                     \\
Home game day       &                     &     106.238\sym{***}\\
                    &                     &    (12.217)         \\
Away game day       &                     &      11.880         \\
                    &                     &     (7.792)         \\
\midrule Effect size [\%]&       21.22         &       21.97         \\
Observations        &      61,172         &      61,172         \\
Region FE           &         yes         &         yes         \\
Time Fe             &         yes         &         yes         \\
Weather Controls    &         yes         &         yes         \\
Holiday FE          &         yes         &         yes         \\
Interact FE         &         yes         &         yes         \\
 
				\bottomrule 
		\end{tabular}}
		\begin{tablenotes} 
			\item \scriptsize \emph{Notes:} The table shows estimates ..... 
			weighted by population shares, se clustered on region-year and year-month, what do FE entail, effect size
			Clustered standard errors are reported in parentheses. \newline Significance levels: * p < 0.10, ** p < 0.05, *** p < 0.01. \newline 	\emph{Source:} xxx.\newline
			\hspace*{15 pt}$^1$: Effect size [\%] corresponds to the coefficient of Home game day.
		\end{tablenotes} 
	\end{threeparttable} 
\end{table}
\vspace*{\fill}\clearpage 

%--------------------------------------------









\newpage
\subsection{Displacement Effects}

Previously I showed evidence that soccer games (home games) increase the assault rate. However, it could very well be that I merely measure a displacement of the offense. It is possible that the increase in violent behavior is offset by a reduction in assaults in other areas or at different times \citep{lindo2018college}. In other words, the assault would have been committed anyway, but just at a different time or at another location. One explanation for that are different population flows around days on which games take place. In the following, I want to investigate the two above mentioned concerns.


% spatial displacement
\textbf{Spatial displacement.--} In order to estimate spatial displacement effects, I investigate the impact of  soccer matches on nearby regions. I define neighboring regions to constitute nearby regions. A neighbor region is a municipality that shares a common boundary with a region that contains a stadium.\footnote{If there are two municipalities that share a common border and each of the regions contains a stadium, both regions will not serve as neighbor regions and they are dismissed from the set of spatial spillover candidates.} Figure \ref{fig_soc_ext:map_neighboring_regions} shows a map of the selected regions. The sample of neighboring regions exhibits a considerably higher number of observations. This is owed to the fact that a region with a stadium has, on average, a bit more than 11 neighbor municipalities. Table \ref{tab_soc_ext:reg_fe_assrate_neighbor_regions} shows the estimates of the impact of a home game on these neighboring regions. In comparison to the baseline effects, the coefficients for spillover effects across space are small and not significantly different from zero. 


% temporal displacement
\textbf{Temporal displacement.--} write stuff here


Interpretation of no displacement effect? 


% table spatial displacement - neighboring regions
%--------------------------------------------
\vspace*{\fill}
\begin{table}[H] \centering 
	\begin{threeparttable} \centering \caption{Spatial displacement effect: Impact on \textbf{Assault rate}}\label{tab_soc_ext:reg_fe_assrate_neighbor_regions}
		{\def\sym#1{\ifmmode^{#1}\else\(^{#1}\)\fi} 
			\begin{tabular}{l*{6}{c}}
				\toprule 
				&\multicolumn{1}{c}{(1)}&\multicolumn{1}{c}{(2)}&\multicolumn{1}{c}{(3)}&\multicolumn{1}{c}{(4)}\\
				\midrule
				Game day            &       5.255         &       5.585         &       6.187         &       7.642         \\
                    &     (5.348)         &     (5.327)         &     (5.384)         &     (5.622)         \\
\midrule Effect size [\%]&        2.58         &        2.74         &        3.03         &        3.75         \\
Observations        &     960,848         &     960,848         &     960,848         &     960,848         \\
Region FE           &         yes         &         yes         &         yes         &         yes         \\
Time Fe             &         yes         &         yes         &         yes         &         yes         \\
Weather Controls    &           -         &         yes         &         yes         &         yes         \\
Holiday FE          &           -         &           -         &         yes         &         yes         \\
Interact FE         &           -         &           -         &           -         &         yes         \\
 
				\bottomrule 
		\end{tabular}}
		\begin{tablenotes} 
			\item \scriptsize \emph{Notes:} The table shows estimates ..... 
			weighted by population shares, se clustered on region-year and year-month, what do FE entail, effect size
			Clustered standard errors are reported in parentheses. \newline Significance levels: * p < 0.10, ** p < 0.05, *** p < 0.01. \newline 	\emph{Source:} xxx.
		\end{tablenotes} 
	\end{threeparttable} 
\end{table}
\vspace*{\fill}\clearpage 

%--------------------------------------------

% map spatial displacement - neighboring regions
\vspace*{\fill}
\begin{figure}[H]\centering
	\caption{Spatial displacement - neighboring regions}\label{fig_soc_ext:map_neighboring_regions}
	\includegraphics[width=0.9\linewidth]{maps/soc_ext_2010_2015_map_stadiums_neighbor_regions_zoom_100.png}
	\begin{minipage}{0.95\linewidth}
		\scriptsize{\emph{Notes:} This map shows the regions that are used in the analysis for spatial displacement effects over the seasons 2010/11 until 2014/15. The neighboring municipalities are chosen to be in the sample for estimating spatial displacement effects if they have a common border with a region that contains a stadium. The red dots are the stadiums, the black outlines indicate federal state boundaries.\newline \emph{Source:} Own representation with data from the Federal Institute for Research on Building, Urban Affairs and Spatial Development (BBSR).}
	\end{minipage}
\end{figure}
\vspace*{\fill}\clearpage










\newpage
\subsection{Robustness}
\textbf{Poisson model of the number of physical assaults.--}First, I test whether my results are sensitive to an alternative measure of the dependent variable. In the following I use the raw number of assaults as the outcome variable. Because of the discrete nature of the dependent variable, and given there are many cells with zero reports, a Poisson model is the model of choice \citep{lindo2018college}. The corresponding regression specification reads as follows:
\begin{align}
E\left[ \text{Assaults}_{rdmy}\right.&|\left.\text{Gameday}_{rdmy},\text{date}_{dmy},\vartheta_r, \mathbf{X}_{rdmy}  \right] \nonumber \\ &=  exp\left( \beta\ \text{Gameday}_{rdmy} + \text{date}_{dmy} + \vartheta_r + \lambda\mathbf{X}_{rdmy} \right) 
\label{eq_soc_ext:poisson_model}
\end{align}

Due to the nonlinearity of the model, Table \ref{tab_soc_ext:assaults_poisson_fixed_effects} shows average marginal effects of the effect of a home game on the number of physical assaults.\footnote{In nonlinear models, coefficients cannot be interpreted as marginal effects. The partial effect for a Poisson model is given by $\frac{\partial E\left[ \text{Assaults}| \mathbf{X}\right]}{\partial x_j} = \beta_j\ exp(\mathbf{x'}\beta)$.} By expanding the model by progressively adding weather controls, holiday fixed effects, and interacted fixed effects, I find stable and robust evidence that a home game leads to an increase in the number of assaults. In column 4, the richest specification, a home game leads to, on average, an increase in the number of assaults by 28.55 percent. This effect is not statistically higher than the baseline results. 




%--------------------------------------------------------------------
% CONCLUSION
%-------------------------------------------------------------------
\bigskip
\section{Concluding remarks}\label{sec_soc_ext:conclusion}




%--------------------------------------------------------------------
% BIBLIOGRAPHY
%--------------------------------------------------------------------
\newpage


\bibliographystyle{ecca_edited}%previous style-chicago
\bibliography{soc_ext_bibliography}

%\printbibliography


%--------------------------------------------------------------------
% FIGURES AND TABLES
%--------------------------------------------------------------------
%\newpage
%\section{Figures and tables}
\newpage
\TODO\section{Figures}
\vspace*{\fill}
{\Huge \begin{center}\textbf{FIGURES}\end{center}}
\vspace*{\fill}\clearpage
%--------------------------------------------





%WMWMWMWMWMWMWMWMWMWMWMWMWMWMWMWMWMWMWMWM
% DATA
%WMWMWMWMWMWMWMWMWMWMWMWMWMWMWMWMWMWMWMWM

%--------------------------------------------
% crime - key distribution
\vspace*{\fill}
\begin{figure}[H]\centering
	\caption{The types of criminal offenses}\label{fig_soc_ext:offense_types_distribution_2014}
	\includegraphics[width=0.9 \linewidth]{descriptive/soc_ext_offense_key_distribution_2014.pdf}
	\begin{minipage}{0.95\linewidth}
		\scriptsize{\emph{Notes:} The figure depicts the frequency distribution of the most common criminal offenses in the Federal Republic of Germany in 2014. The most common offense type is intentional simple bodily injury (224000), followed by threats (232300), and two forms of dangerous and serious bodily injury (222110 \& 222010). These four offense types together comprise around 75\% of all criminal offenses. \newline \emph{Source:} Own representation with data from the Federal Criminal Police Office.}
	\end{minipage}
\end{figure}
\vspace*{\fill}\clearpage
%--------------------------------------------

% crime - assault distribution over time
\vspace*{\fill}
\begin{figure}[H]\centering
	\caption{Distribution of assaults across time}\label{fig_soc_ext:assault_time_distribution_2014}
	\begin{subfigure}[h]{0.48\linewidth}\centering
		\includegraphics[width=\linewidth]{descriptive/soc_ext_assaults_per_hour.pdf}
	\end{subfigure}
	\begin{subfigure}[h]{0.48\linewidth}\centering
		\includegraphics[width=\linewidth]{descriptive/soc_ext_assaults_per_dow.pdf}
	\end{subfigure}

	\begin{subfigure}[h]{0.48\linewidth}\centering
		\includegraphics[width=\linewidth]{descriptive/soc_ext_assaults_per_month_2014.pdf}
	\end{subfigure}
	\begin{subfigure}[h]{0.48\linewidth}\centering
		\includegraphics[width=\linewidth]{descriptive/soc_ext_assaults_per_day_2014_single.pdf}
	\end{subfigure}
	\begin{minipage}{\linewidth}
		\scriptsize{\emph{Notes:} The figure shows the distribution of assaults across the course of a day, days of the week, across months (adjusted for the number of days per month), and across the days of the year in the Federal Republic of Germany.\newline \emph{Source:} Own representation with data from the Federal Criminal Police Office.\newline \hspace{1 em} $^1$ The figures in panel C and D are solely based on the year 2014. Please consult the appendix for figures from the other years.}
	\end{minipage}
\end{figure}
\vspace*{\fill}\clearpage

%--------------------------------------------


% MAP: STADIUMS
\vspace*{\fill}
\begin{figure}[H]\centering
	\caption{The stadiums}\label{fig_soc_ext:map_gem_stadiums}
	\includegraphics[width=0.9\linewidth]{maps/soc_ext_2010_2015_Gemeinde_stadiums_75.png}
	\begin{minipage}{0.95\linewidth}
		\scriptsize{\emph{Notes:} This map shows the stadiums used in the analysis over the seasons 2010/11 until 2014/15. The black outlines indicate federal state boundaries.\newline \emph{Source:} Own representation with data from the Federal Institute for Research on Building, Urban Affairs and Spatial Development (BBSR).}
	\end{minipage}
\end{figure}
\vspace*{\fill}\clearpage




%--------------------------------------------
% Descriptives about soccer matches
\vspace*{\fill}
\begin{figure}[H]\centering
	\caption{Soccer matches}\label{fig_soc_ext:descriptives_matches_time_attendance}
	\includegraphics[width=0.9 \linewidth]{descriptive/soc_ect_desc_soccer_macthes_attendance.pdf}	
	\begin{minipage}{\linewidth}
		\scriptsize{\emph{Notes:} The figures show key aspects of soccer games in the data set. Panel A shows how the number of matches vary over the course of a week, Panel B plots the distribution of matches over the course of a day, and Panel C shows kernel densities for the number of spectators (in thousand) across the three leagues.\newline \emph{Source:} Own representation with data obtained from kicker.de.}
	\end{minipage}
\end{figure}
\vspace*{\fill}\clearpage




%--------------------------------------------

% MAP: Stadium and Weather monitors
\vspace*{\fill}
\begin{figure}[H]\centering
	\caption{The stadiums and their closest weather monitors}\label{fig_soc_ext:map_gem_stadiums_nearest_monitor}
	\includegraphics[width=0.9 \linewidth]{maps/soc_ext_2010_2015_Gemeinde_stadiums_nearest_monitor_75.png}
	\begin{minipage}{0.95\linewidth}
		\scriptsize{\emph{Notes:} This map shows the stadiums used in the analysis over the seasons 2010/11 until 2014/15 (red dots) and the closest weather monitors (blue dots). The orange lines indicate how the weather monitors are assigned to the stadiums. The black outlines indicate federal state boundaries.\newline \emph{Source:} Own representation with data from the Federal Institute for Research on Building, Urban Affairs and Spatial Development (BBSR).}
	\end{minipage}
\end{figure}
\vspace*{\fill}\clearpage
%--------------------------------------------




% WEATHER MONITORS: histogram of the distance stadium <-> monitor
\vspace*{\fill}
\begin{figure}[H]\centering
	\caption{The distance between stadiums and their closest weather monitor}\label{fig_soc_ext:hist_distance_stadium_monitors}
	\includegraphics[width=0.9 \linewidth]{descriptive/soc_ext_distance_monitors_stadiums.pdf}
	\begin{minipage}{0.95\linewidth}
		\scriptsize{\emph{Notes:} The figure shows the distribution of the distance between stadiums and their closest weather monitor. The average distance between the two is 15 km. \newline \emph{Source:} Own representation with data from Germany's National Meterological Service (Deutscher Wetterdienst).}
	\end{minipage}
\end{figure}
\vspace*{\fill}\clearpage
%--------------------------------------------
% To be put in: 
%	- histogram: distance weather monitors to stadiums
%	- barplot: number of offences, by key


%--------------------------------------------

%WMWMWMWMWMWMWMWMWMWMWMWMWMWMWMWMWMWMWMWM
% RESULTS
%WMWMWMWMWMWMWMWMWMWMWMWMWMWMWMWMWMWMWMWM


\vspace*{\fill}
\begin{figure}[H]\centering
	\caption{The average assault rate on gamedays and days where no game takes place}\label{fig_soc_ext:}
	\includegraphics[width=0.9 \linewidth]{descriptive/soc_ext_descr_assaultrate_gd_nongd_hor.pdf}
	\begin{minipage}{0.95\linewidth}
		\scriptsize{\emph{Notes:} }
	\end{minipage}
\end{figure}
\vspace*{\fill}\clearpage






\begin{landscape}
	\vspace*{\fill}
	\begin{figure}[H]\centering
		\caption{The age profile of the effect}\label{fig_soc_ext:fe_age_profile}
		\begin{subfigure}[h]{0.31\linewidth}\centering\caption{Total}
			\includegraphics[width=\linewidth]{regression/fe_agebrackets_t.pdf}
		\end{subfigure}
		\begin{subfigure}[h]{0.31\linewidth}\centering\caption{Women}
			\includegraphics[width=\linewidth]{regression/fe_agebrackets_f.pdf}
		\end{subfigure}
		\begin{subfigure}[h]{0.31\linewidth}\centering\caption{Men}
			\includegraphics[width=\linewidth]{regression/fe_agebrackets_m.pdf}
		\end{subfigure}
		\scriptsize
		\begin{minipage}{\linewidth}
			\emph{Notes:} The figures show FE estimates with the richest specification along with 95\% confidence intervals for all assaults, and for both genders separately. 
		\end{minipage}
	\end{figure}
	\vspace*{\fill}\clearpage
\end{landscape}



%--------------------------------------------------------------------
% TABLES
%--------------------------------------------------------------------
\newpage
\TODO\section{Tables}
\vspace*{\fill}
{\Huge \begin{center}\textbf{TABLES}\end{center}}
\vspace*{\fill}\clearpage


%--------------------------------------------
\vspace*{\fill}
\begin{table}[H] \centering 
	\begin{threeparttable} \centering \caption{Effects on \textbf{Assault rate}}\label{tab_soc_ext:reg_fe_assrate}
		{\def\sym#1{\ifmmode^{#1}\else\(^{#1}\)\fi} 
			\begin{tabular}{l*{6}{c}}
				\toprule 
				&\multicolumn{1}{c}{(1)}&\multicolumn{1}{c}{(2)}&\multicolumn{1}{c}{(3)}&\multicolumn{1}{c}{(4)}\\
				\midrule
				Game day            &       2.740\sym{***}&       2.766\sym{***}&       2.813\sym{***}&       2.677\sym{***}\\
                    &     (0.319)         &     (0.312)         &     (0.313)         &     (0.284)         \\
\midrule Effect size [\%]&       22.00         &       22.21         &       22.59         &       21.50         \\
Observations        &      88,028         &      88,028         &      88,028         &      88,028         \\
Region FE           &         yes         &         yes         &         yes         &         yes         \\
Date Fe             &         yes         &         yes         &         yes         &         yes         \\
Weather Controls    &           -         &         yes         &         yes         &         yes         \\
Holiday FE          &           -         &           -         &         yes         &         yes         \\
Interact FE         &           -         &           -         &           -         &         yes         \\
 
				\bottomrule 
		\end{tabular}}
		\begin{tablenotes} 
			\item \scriptsize \emph{Notes:} The table shows estimates ..... 
			weighted by population shares, se clustered on region-year and year-month, what do FE entail, effect size
			Clustered standard errors are reported in parentheses. \newline Significance levels: * p < 0.10, ** p < 0.05, *** p < 0.01. \newline 	\emph{Source:} xxx.
		\end{tablenotes} 
	\end{threeparttable} 
\end{table}
\vspace*{\fill}\clearpage 

%--------------------------------------------

\vspace*{\fill}
\begin{table}[H] \centering 
	\begin{threeparttable} \centering \caption{Effects on \textbf{Assault rate with leads/lags}}\label{tab_soc_ext:reg_fe_assrate_leads_lags}
		{\def\sym#1{\ifmmode^{#1}\else\(^{#1}\)\fi} 
			\begin{tabular}{l*{6}{c}}
				\toprule 
				&\multicolumn{1}{c}{(1)}&\multicolumn{1}{c}{(2)}&\multicolumn{1}{c}{(3)}&\multicolumn{1}{c}{(4)}\\
				\midrule
				Game day            &       2.777\sym{***}&       2.810\sym{***}&       2.857\sym{***}&       2.770\sym{***}\\
                    &     (0.346)         &     (0.338)         &     (0.336)         &     (0.300)         \\
Day after game      &      -0.105         &      -0.076         &      -0.065         &       0.279         \\
                    &     (0.189)         &     (0.190)         &     (0.192)         &     (0.182)         \\
Day before game     &       0.287         &       0.318         &       0.339         &       0.352\sym{*}  \\
                    &     (0.241)         &     (0.238)         &     (0.229)         &     (0.206)         \\
\midrule Effect size [\%]&       22.34         &       22.60         &       22.98         &       22.28         \\
Observations        &      87,438         &      87,438         &      87,438         &      87,438         \\
Region FE           &         yes         &         yes         &         yes         &         yes         \\
Date Fe             &         yes         &         yes         &         yes         &         yes         \\
Weather Controls    &           -         &         yes         &         yes         &         yes         \\
Holiday FE          &           -         &           -         &         yes         &         yes         \\
Interact FE         &           -         &           -         &           -         &         yes         \\
 
				\bottomrule 
		\end{tabular}}
		\begin{tablenotes} 
			\item \scriptsize \emph{Notes:} The table shows estimates ..... 
			weighted by population shares, se clustered on region-year and year-month, what do FE entail, effect size
			Clustered standard errors are reported in parentheses. \newline Significance levels: * p < 0.10, ** p < 0.05, *** p < 0.01. \newline 	\emph{Source:} xxx.
		\end{tablenotes} 
	\end{threeparttable} 
\end{table}
\vspace*{\fill}\clearpage 

%--------------------------------------------
% subcategories: VICTIM characteristics
\begin{landscape}
	\vspace*{\fill}
	\begin{table}[H] \centering 
		\begin{threeparttable} \centering \caption{Effects on Assault rate, by victim characteristics}
			\label{tab_soc_ext:reg_fe_assrate_victim_chars}
			{\def\sym#1{\ifmmode^{#1}\else\(^{#1}\)\fi} 
				\begin{tabular}{l*{6}{c}}
					\toprule 
					&\multicolumn{1}{c}{(1)}&\multicolumn{1}{c}{(2)}&\multicolumn{1}{c}{(3)}&\multicolumn{1}{c}{(4)}&\multicolumn{1}{c}{(5)}&\multicolumn{1}{c}{(6)}\\
					& & \multicolumn{2}{c}{Gender} & \multicolumn{3}{c}{Victim-suspect-relationship} \\
					\cmidrule(lr){3-4} \cmidrule(lr){5-7}
					& baseline & women & men & strangers & \clb{c}{prior\\relationship}  & domestic \\
					\midrule
					Game day            &       2.677\sym{***}&       0.245\sym{***}&       2.432\sym{***}&       1.939\sym{***}&       0.738\sym{***}&       0.096\sym{**} &       0.434\sym{***}\\
                    &     (0.284)         &     (0.090)         &     (0.232)         &     (0.207)         &     (0.116)         &     (0.045)         &     (0.078)         \\
\midrule Effect size [\%]&       21.50         &        4.99         &       32.27         &       40.43         &        9.64         &        5.40         &       96.98         \\
Observations        &      88,028         &      88,028         &      88,028         &      88,028         &      88,028         &      88,028         &      88,028         \\
Region FE           &         \checkmark         &         \checkmark         &         \checkmark         &         \checkmark         &         \checkmark         &         \checkmark         &         \checkmark         \\
Date Fe             &         \checkmark         &         \checkmark         &         \checkmark         &         \checkmark         &         \checkmark         &         \checkmark         &         \checkmark         \\
Weather Controls    &         \checkmark         &         \checkmark         &         \checkmark         &         \checkmark         &         \checkmark         &         \checkmark         &         \checkmark         \\
Holiday FE          &         \checkmark         &         \checkmark         &         \checkmark         &         \checkmark         &         \checkmark         &         \checkmark         &         \checkmark         \\
Interact FE         &         \checkmark         &         \checkmark         &         \checkmark         &         \checkmark         &         \checkmark         &         \checkmark         &         \checkmark         \\
 
					\bottomrule 
			\end{tabular}}
			\begin{tablenotes} 
				\item \scriptsize \emph{Notes:} The table shows estimates ..... 
				weighted by population shares, se clustered on region-year and year-month, what do FE entail, effect size, strangers and prior relationship are defined in formal way (e.g. what type of family, acquaintance - if any), whereas the domestic sphere is defined in a spatial-social manner (e.g. living in the same household, education or care contexts). 
				Clustered standard errors are reported in parentheses. \newline Significance levels: * p < 0.10, ** p < 0.05, *** p < 0.01. \newline 	\emph{Source:} xxx.
			\end{tablenotes} 
		\end{threeparttable} 
	\end{table}
	\vspace*{\fill}\clearpage 
\end{landscape}
%--------------------------------------------
% subcategories: CRIME characteristics
\begin{landscape}
	\vspace*{\fill}
	\begin{table}[H] \centering 
		\begin{threeparttable} \centering \caption{Effects on Assault rate, by crime characteristics}
			\label{tab_soc_ext:reg_fe_assrate_crime_chars}
			{\def\sym#1{\ifmmode^{#1}\else\(^{#1}\)\fi} 
				\begin{tabular}{l*{7}{c}}
					\toprule 
					&\multicolumn{1}{c}{(1)}&\multicolumn{1}{c}{(2)}&\multicolumn{1}{c}{(3)}&\multicolumn{1}{c}{(4)}&\multicolumn{1}{c}{(5)}&\multicolumn{1}{c}{(6)} &\multicolumn{1}{c}{(7)}\\
					& & \multicolumn{4}{c}{\textbf{Timing}} & \multicolumn{2}{c}{\textbf{assault execution}} \\
					\cmidrule(lr){3-6} \cmidrule(lr){7-8}
					& baseline & spring & summer & fall & winter & attempt & success \\
					\midrule
					Game day            &      97.744\sym{***}&     102.860\sym{***}&      54.066\sym{**} &      90.477\sym{***}&     104.015\sym{***}&      13.406\sym{***}&      84.339\sym{***}\\
                    &    (10.370)         &    (14.765)         &    (17.666)         &    (13.583)         &    (19.518)         &     (2.198)         &     (8.689)         \\
\midrule Effect size [\%]&       21.49         &       20.99         &       13.38         &       20.70         &       22.50         &       56.20         &       19.57         \\
Observations        &      88,028         &      27,140         &      14,632         &      21,476         &      24,780         &      88,028         &      88,028         \\
Region FE           &         yes         &         yes         &         yes         &         yes         &         yes         &         yes         &         yes         \\
Time Fe             &         yes         &         yes         &         yes         &         yes         &         yes         &         yes         &         yes         \\
Weather Controls    &         yes         &         yes         &         yes         &         yes         &         yes         &         yes         &         yes         \\
Holiday FE          &         yes         &         yes         &         yes         &         yes         &         yes         &         yes         &         yes         \\
Interact FE         &         yes         &         yes         &         yes         &         yes         &         yes         &         yes         &         yes         \\
 
					\bottomrule 
			\end{tabular}}
			\begin{tablenotes} 
				\item \scriptsize \emph{Notes:} The table shows estimates ..... 
				weighted by population shares, se clustered on region-year and year-month, what do FE entail, effect size
				Clustered standard errors are reported in parentheses. \newline Significance levels: * p < 0.10, ** p < 0.05, *** p < 0.01. \newline 	\emph{Source:} xxx.
			\end{tablenotes} 
		\end{threeparttable} 
	\end{table}
	\vspace*{\fill}\clearpage 
\end{landscape}
%--------------------------------------------
% robustensss: ASSAULTS and POISSON
\vspace*{\fill}
\begin{table}[H] \centering 
	\begin{threeparttable} \centering \caption{Effects on \textbf{the number of Assaults}}\label{tab_soc_ext:assaults_poisson_fixed_effects}
		{\def\sym#1{\ifmmode^{#1}\else\(^{#1}\)\fi} 
			\begin{tabular}{l*{6}{c}}
				\toprule 
				&\multicolumn{1}{c}{(1)}&\multicolumn{1}{c}{(2)}&\multicolumn{1}{c}{(3)}&\multicolumn{1}{c}{(4)}\\
				\midrule
				Game day            &       1.479\sym{***}&       1.527\sym{***}&       1.575\sym{***}&       1.483\sym{***}\\
                    &     (0.269)         &     (0.287)         &     (0.286)         &     (0.238)         \\
\midrule Effect size [\%]&       28.66         &       29.58         &       30.53         &       28.55         \\
Observations        &      88,028         &      88,028         &      88,028         &      87,475         \\
Region FE           &         yes         &         yes         &         yes         &         yes         \\
Date Fe             &         yes         &         yes         &         yes         &         yes         \\
Weather Controls    &           -         &         yes         &         yes         &         yes         \\
Holiday FE          &           -         &           -         &         yes         &         yes         \\
Interact FE         &           -         &           -         &           -         &         yes         \\
 
				\bottomrule 
		\end{tabular}}
		\begin{tablenotes} 
			\item \scriptsize \emph{Notes:} The table shows average marginal effects of the effect of gd of the number of physical assaults xxx 
			weighted by population shares, se clustered on region-year and year-month, what do FE entail, effect size
			Clustered standard errors are reported in parentheses. \newline Significance levels: * p < 0.10, ** p < 0.05, *** p < 0.01. \newline 	\emph{Source:} xxx.
		\end{tablenotes} 
	\end{threeparttable} 
\end{table}
\vspace*{\fill}\clearpage 




%--------------------------------------------------------------------
% APPENDIX
%--------------------------------------------------------------------
\newpage
\TODO\section{Appendix}
\vspace*{\fill}
{\Huge \begin{center}\textbf{APPENDIX}\end{center}}
\vspace*{\fill}\clearpage


\renewcommand\thefigure{A\arabic{figure}}
\setcounter{figure}{0} 
\captionsetup[subfigure]{labelformat=parens}



%--------------------------------------------
% crime - assault distribution over time
\vspace*{\fill}
\begin{figure}[H]\centering
	\caption{Distribution of assaults across days of the year, per year}\label{fig_soc_ext:assault_year_distribution}
	\begin{subfigure}[h]{0.48\linewidth}\centering
		\includegraphics[width=\linewidth]{descriptive/soc_ext_assaults_per_day_2011.pdf}
	\end{subfigure}
	\begin{subfigure}[h]{0.48\linewidth}\centering
		\includegraphics[width=\linewidth]{descriptive/soc_ext_assaults_per_day_2012.pdf}
	\end{subfigure}\begin{subfigure}[h]{0.48\linewidth}\centering
		\includegraphics[width=\linewidth]{descriptive/soc_ext_assaults_per_day_2013.pdf}
	\end{subfigure}
	\begin{subfigure}[h]{0.48\linewidth}\centering
		\includegraphics[width=\linewidth]{descriptive/soc_ext_assaults_per_day_2014.pdf}
	\end{subfigure}
	\begin{subfigure}[h]{0.48\linewidth}\centering
		\includegraphics[width=\linewidth]{descriptive/soc_ext_assaults_per_day_2015.pdf}
	\end{subfigure}
	\begin{minipage}{\linewidth}
		\scriptsize{\emph{Notes:} The figure shows the relative frequency of assaults per day between 2011 and 2015 in the Federal Republic of Germany.\newline \emph{Source:} Own representation with data from the Federal Criminal Police Office}
	\end{minipage}
\end{figure}
\vspace*{\fill}\clearpage
%--------------------------------------------




\end{document}























%--------------------------------------------


%% Attendance across leagues (violin plots)
%\vspace*{\fill}
%\begin{figure}[H]\centering
%	\caption{Attendance across the leagues}\label{fig_soc_ext:violin_attendance_per_league}
%	\includegraphics[width=0.9 \linewidth]{descriptive/soc_ext_attendance_per_league.pdf}
%	\begin{minipage}{0.95\linewidth}
%		\scriptsize{\emph{Notes:} The figure shows the distributions of the number of spectators (in thousand) per league. The white dot represents the median and the thick bar in the center depicts the interquartile range. The median number of viewers are 42,000 for league one, 15,000 for league two, and 5,000 for league three. \newline \emph{Source:} Own representation with data obtained from kicker.de.}
%	\end{minipage}
%\end{figure}
%\vspace*{\fill}\clearpage